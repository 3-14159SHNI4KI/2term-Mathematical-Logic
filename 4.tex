\subsection*{Логическая равносильность формул}
\begin{definition}
    Формулы $\Phi$, $\Uppsi$ называются \textit{логически равносильными} (или просто \textit{равносильными}), если они принимают одинаковые логические значения при любых истинностных значениях их переменных.

    Это равносильно условию $\vDash\Phi\Leftrightarrow\Uppsi$.
\end{definition}

\begin{definition}
    Для обозначения логически эквивалентных формул используется символическая запись $\Phi = \Uppsi$, или $\Phi \cong \Uppsi$.

    Такие выражения называют \textit{логическими равенствами} или просто \textit{равенствами} формул.
\end{definition}

\begin{lemma}[1]
    Справедливы следующие равенства формул:
    \begin{enumerate}
        \item Свойства ассоциативности дизъюнкции и конъюнкции:
        
        $X\lor(Y\lor Z) = (X\lor Y)\lor Z$

        $X\land (Y\land Z) = (X\land Y)\land Z$

        \item Свойства коммутативности дизъюнкции и конъюнкции:
        
        $X\lor Y = Y\lor X$

        $X\land Y = Y\land X$

        \item Свойства идемпотентности дизъюнкции и конъюнкции:
        
        $X\lor X = X$

        $X\land X = X$

        \item Законы дистрибутивности конъюнкции относительно дизъюнкции и дизъюнкции относительно конъюнкции:
        
        $X\land(Y\lor Z) = (X\land Y)\lor(X\land Z)$

        $X\lor(Y\land Z) = (X\lor Y)\land(X\lor Z)$

        \item Законы де Моргана:
        
        $\lnot(X\land Y) = \lnot X \lor \lnot Y$

        $\lnot(X\lor Y) = \lnot X \land \lnot Y$

        \item Законы поглощения:
        
        $(X\land Y)\lor X = X$

        $(X\lor Y)\land X = X$

        \item Закон двойного отрицания:
        
        $\lnot\lnot X = X$

        \item Взаимосвязь импликации с дизъюнкцией и конъюнкцией:
        
        $X\Rightarrow Y = \lnot X \lor Y$

        $X\Rightarrow Y = \lnot(X\land \lnot Y)$

        \item Взаимосвязь эквивалентности с импликацией, дизъюнкцией и конъюнкцией:
        
        $X\Leftrightarrow Y = (X\Rightarrow Y)\land(Y\Rightarrow X)$

        $X\Leftrightarrow Y = (\lnot X \lor Y) \land (X\lor \lnot Y)$
    \end{enumerate}
\end{lemma}

\subsection*{Равносильные преобразования формул}

\begin{lemma}[Правило замены] 
    Если формулы $\Phi$, $\Phi '$ равносильны, то для любой формулы $\Uppsi(X)$, содержащей переменную $X$, выполняется равенство: $\Uppsi(\Phi) = \Uppsi(\Phi')$.
\end{lemma}

Это правило означает, что при замене в любой формуле $\Uppsi = \Uppsi(\Phi)$ некоторой ее подформулы $\Phi$ на равносильную ей формулу $\Phi'$ получается формула $\Uppsi = \Uppsi(\Phi')$, равносильная исходной формуле $\Uppsi$.

Такие переходы называются \textit{равносильными преобразованиями формул}.

\begin{example}
    Формула $\Phi = (X\Rightarrow Y) \Rightarrow Z$ с помощью равенств 5, 7, 8 из леммы 1 равносильно преобразовывается следующим образом:

    $\Phi = (X\Rightarrow Y)\Rightarrow Z = \lnot(X\Rightarrow Y)\lor Z = \lnot(\lnot(X\land\lnot Y))\lor Z = (X\land \lnot Y)\lor Z$.
\end{example}
