\subsection*{Интерпретации формул алгебры предикатов}
\begin{definition}
    \textit{Область интерпретации} "--- непустое множество $M$, которое является областью возможных значений всех предметных переменных.
    
    $n-$местным предикатным символам $P$ присваиваются конкретные значения $P_M$ $n-$местных предикатов на множестве $M$.

    Соответствие $\beta : P \mapsto P_M$ называется \textit{интерпретацией предикатных символов}.

    Область интерпретации $M$ вместе с интерпретацией предикатных символов $\beta$ называется \textit{интерпретацией формул алгебры предикатов} и обозначается $(M,\beta)$ или просто $M$.
\end{definition}

\begin{definition}
    При наличии интепретации $M$ конкретные значения предметным переменным формул алгебры предикатов присваиваются с помощью отображения $a$ множества всех предметных переменных $X$ в область интерпретации $M$.

    Такие отображения называются \textit{оценками} предметных переменных.
\end{definition}

\begin{definition}
    \textit{Выполнимость формулы} $\Phi$ в интерпретации $M$ при оценке $a$ обозначается $M\models_a \Phi$ "--- читается <<формула $\Phi$ истинна в интерпретации $M$ при оценке $a$>> и определяется следующим образом:
    \begin{enumerate}
        \item Если $\Phi=P(x_1,\ldots,x_n)$ для $n-$местного предикатного символа $P$ и предметных переменных $x_1,\ldots,x_n$, то $M\models_a\Phi$ тогда и только тогда, когда высказывание $P_M(a(x_1),\ldots,a(x_n))$ истинно.
        
        \item Если $\Phi=\lnot \Uppsi$ для формулы $\Uppsi$, то $M\models_a\Phi$ тогда и только тогда, когда неверно, что $M\models_a\Uppsi$.

        \item Если $\Phi=\Phi_1\land\Phi_2$ для формул $\Phi_1,\Phi_2$, то $M\models_a\Phi$ тогда и только тогда, когда $M\models_a\Phi_1$ и $M\models_a\Phi_2$.

        \item Если $\Phi = \Phi_1\lor\Phi_2$ для формул $Phi_1,Phi_2$, то $M\models_a\Phi$ тогда и только тогда, когда $M\models_a\Phi_1$ или $M\models_a\Phi_2$.

        \item  Если $\Phi = \Phi_1\Rightarrow\Phi_2$ для формул $\Phi_1,\Phi_2$, то $M\models_a\Phi$ тогда и только тогда, когда неверно, что $M\models_a\Phi_1$ и $M\models_a\lnot\Phi_2$.
        
        \item Если $\Phi=\Phi_1\Leftrightarrow\Phi_2$ для формул $\Phi_1,\Phi_2$, то $M\models_a\Phi$ тогда и только тогда, когда $M\models_a\Phi_1$, $M\models_a\Phi_2$ одновременно верны или нет.

        \item Если $\Phi=(\forall x)\Uppsi$ для некоторой формулы $\Uppsi$, то $M\models_a\Phi$ тогда и только тогда, когда $M\models_{a'}\Uppsi$ для всех оценок $a'$, отличающихся от оценки $a$ возможно только на элементе $x$.

        \item Если $\Phi=(\exists x)\Uppsi$ для некоторой формулы $\Uppsi$, то $M\models_a\Phi$ тогда и только тогда, когда $M\models_{a'}\Uppsi$ для некоторой оценки $a'$, отличающейся от оценки $a$ возможно только на элементе $x$.
    \end{enumerate}
\end{definition}

\begin{example}
    $M\models_a P(x)\Leftrightarrow Q(x)$ равносильно истинности высказывания $$P_M(a(x))\Leftrightarrow Q_M(a(x))$$
\end{example}

\begin{example}
    $M\models_a P(x)\Rightarrow Q(x)$ равносильно истинности высказывания $$P_M(a(x))\Rightarrow Q_M(a(x))$$
\end{example}

\subsection*{Классификация формул алгебры предикатов}
\begin{definition}
    В интепретации $M$ формула $\Phi$ называется:
    \begin{itemize}
        \item \textit{Общезначимой} или (\textit{тождественно истинной}), если $M\models_a\Phi$ при любых оценках $a$

        \item \textit{Выполнимой}, если $M\models_a\Phi$ для некоторой оценки $a$

        \item \textit{Опровержимой}, если для некоторой оценки $a$ неверно, что $M\models_a\Phi$

        \item \textit{Тождественно ложной}, если для любой оценки $a$ неверно, что $M\models_a\Phi$
    \end{itemize}
\end{definition}

Формула $\Phi$ \textit{общезначима} в интепретации $M$ (с интерпретаций $P_M$ $n-$арных предикатных символов $P$), если она превращается в тождественно истинный на множестве $M$ предикат.

Символическая запись $M\models \Phi$.

Формула $\Phi$ в интерпретации $M$ \textit{выполнима, опровержима} или \textit{тождественно ложна}, если она превращается соответственно в выполнимый, опровержимый или тождественно ложный на множестве $M$ предикат $P_M$.

\begin{example}
    $M\models_a P(x)\Leftrightarrow Q(x)$ равносильно $P_M(a(x))\Leftrightarrow Q_M(a(x))$

    $M\models_a P(x)\Rightarrow Q(x)$ равносильно $P_M(a(x))\Rightarrow Q_M(a(x))$

    $M\models P(x)\Leftrightarrow Q(x)$ равносильно $P_M^+ = Q_M^+$

    $M\models P(x)\Rightarrow Q(x)$ равносильно $P_M^+ \subset  Q_M^+$

    $M\models(\forall x)P(x)$ равносильно $P_M^+ = M$

    $M\models (\exists x)P(x)$ равносильно $P_M^+ \neq \varnothing $
\end{example}

\begin{definition}
    Формула $\Phi$ называется \textit{тождественно истинной}, если она тождественно истинна в любой интепретации $M$. Такая формула называется также \textit{общезначимой формулой}, или \textit{тавтологией алгебры предикатов} и обозначается $\models\Phi$.

    Множество всех тавтологий алгебры предикатов обозначим 
    $\mathscr{T_{\text{АП}}}$.

\end{definition}

\begin{definition} 
    Формула $\Phi$ называется \textit{тождественно ложной} или \textit{противоречием}, если она тождественно ложна в любой интерпретации $M$.

    По определению противоречивость формулы $\Phi$ равносильна условию $\models\lnot\Phi$.
\end{definition}

\begin{definition}
    Формула $\Phi$ называется \textit{выполнимой}, если она выполнима хотя в одной интепретации $M$, которая называется \textit{моделью} этой формулы.
\end{definition}

Таким образом, получаем определение 44.

\begin{remark}
    Если формула $\Phi$ является предложением, то она не содержит свободныъ вхождений переменных и, следовательно, не зависит от оценок $a$ предметных переменных в области интерпретации $M$.

    Значит, предложение $\Phi$ в интепретации $M$ общезначимо в том и только том случае, если оно выполнимо (т.е. выполняется хотя бы при одной оценке $a$ предметных переменных в области интепретации $M$).
\end{remark}