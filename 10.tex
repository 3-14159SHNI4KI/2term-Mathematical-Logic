\begin{definition}
    \textit{Алгеброй предикатов} называется множество всех предикатов $\mathscr{P}$ с логическими операциями $\lnot,\land,\lor,\Rightarrow,\Leftrightarrow$ и операциями квантификации $(\forall x),(\exists x)$ для всех предметныъ переменных $x$.
\end{definition}

Свойства алгебры предикатов $\mathscr{P}$ описываются с помощью специальных формул, которые строятся из символов предикатов и предметных переменных с помощью специальных вспомогательных символов "--- скобок и знаков логических операций над предикатами. 

\textit{Алфавит} алгебры предикатов состоит из следующих символов:


1. \textit{Предметные переменные} $x_1,x_2,\ldots$, которые используются для обозначения элементов множества допустимых значений

2. $n-$местные \textit{предикатные символы} $P,Q,\ldots$, которые используются для обозначения $n-$местных предикатов на множестве допустимых значений

3. Символы логических операций $\lnot,\land,\lor,\Rightarrow,\Leftrightarrow,\forall,\exists$

4. Вспомогательные символы $(,)$ и другие


\begin{definition}
    \textit{Формулы} алгебры предикатов определяются по индукции следующим образом:
    \begin{enumerate}
        \item Для любого $n-$местного предикатного символа $P$ и любых $n$ предметных переменных $x_1,\ldots,x_n$ выражение $P(x_1,\ldots,x_n)$ есть формула, которая называется \textit{элементарной} (или \textit{атомарной}) \textit{формулой}.
        \item Если $\Phi,\Uppsi - $ формулы, то формулами являются также выражения $(\lnot\Phi)$, $(\Phi\land\Uppsi)$, $(\Phi\lor\Uppsi)$, $(\Phi\Rightarrow\Uppsi)$, $(\Phi\Leftrightarrow\Uppsi)$.
        \item Если $\Phi - $ формула и $x - $ предметная переменная, то формулами являются такжее выражения $(\forall x)\Phi$, $(\exists x)\Phi$; при этом переменная $x$ и формула $\Phi$ называются \textit{областью действия} соответствующего \textit{квантора}.
    \end{enumerate}
\end{definition}

\begin{definition}
    Если в формулу $\Phi$ входят переменные $x_1,\ldots,x_n$, то записывают $\Phi = \Phi(x_1,\ldots,x_n)$.

    Вхождение предметной переменной $x$ в формулу $\Phi$ называется \textit{связным}, если она находится в области действия одного из кванторов по этой переменной; в противном случае вхождение предметной переменной $x$ в формулу $\Phi$ называется \textit{свободным}.    

    Формула без свободных вхождений переменных называется \textit{замкнутой формулой} или \textit{предложением}.

    Фактически формуле определяет предикат с переменными, которые входят в формулу свободно.
\end{definition}

