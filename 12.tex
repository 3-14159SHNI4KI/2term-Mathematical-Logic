%Логическое равенство формул алгебры предикатов. Свойства логических операций над предикатами
%логическое следование
\subsection*{Тавтологии алгебры предикатов}
Любая тавтология алгебры высказываний 
является тавтологией алгебры предикатов. 
Более того, тавтологии алгебры высказываний 
дают возможность легко получать тавтологии 
алгебры предикатов с помощью следующего 
очевидного результата.

\begin{lemma}[1]
    Если $\Phi(X_1,\ldots,X_n)$ "--- тавтология 
алгебры высказываний, то для любых формул 
алгебры предикатов $\Phi_1,\ldots,\Phi_n$ формула $\Phi(\Phi_1,\ldots,\Phi_n)$
является тавтологией алгебры предикатов.

С другой стороны, в алгебре предикатов можно 
получить много принципиально новых тавтологий с 
помощью следующих свойств кванторов.
\end{lemma}

\begin{lemma}[2]
    Для любых формул $\Phi,\Uppsi$ следующие 
формулы являются тавтологиями:

\begin{enumerate}
    \item $\lnot(\forall x)\Phi \Leftrightarrow (\exists x)\lnot\Phi$ \qquad
    $\lnot(\exists x)\Phi\Leftrightarrow (\forall x)\lnot\Phi$ 

    $(\forall x)\Phi\Leftrightarrow\lnot(\exists x)\lnot\Phi$ \qquad
    $(\exists x)\Phi\Leftrightarrow\lnot(\forall x)\lnot\Phi$


    \item $(\forall x)(\forall y)\Phi\Leftrightarrow(\forall y)(\forall x)\Phi$
    
    $(\exists x)(\forall y)\Phi\Rightarrow(\forall y)(\exists x)\Phi$


    \item $(\forall x)(\Phi\land\Uppsi)\Leftrightarrow(\forall x)\Phi\land(\forall x)\Uppsi$
    
    $(\exists x)(\Phi\lor\Uppsi)\Leftrightarrow(\exists x)\Phi\lor(\exists x)\Uppsi$


    \item $(\forall x)(\Phi \pi \Uppsi)\Leftrightarrow(\forall x)\Phi\pi\Uppsi$, где $\pi$ -- символ одной из операций $\land,\lor$
    
    \item $(\exists x)(\Phi\pi\Uppsi)\Leftrightarrow(\exists x)\Phi\pi\Uppsi$, где $\pi$ -- символ одной из операций $\land,\lor$
    
    если в формулу $\Uppsi$ предметная переменная $x$ не входит свободно, а также

    \item $(\forall x)(\Phi\Rightarrow\Uppsi)\Leftrightarrow((\exists x)\Phi\Rightarrow\Uppsi)$
    
    $(\exists x)(\Phi\Rightarrow\Uppsi)\Leftrightarrow((\forall x)\Phi\Rightarrow\Uppsi)$

    \item $(\forall x)(\Uppsi\Rightarrow\Phi)\Leftrightarrow(\Uppsi\Rightarrow(\forall x)\Phi)$
    
    $(\exists x)(\Uppsi\Rightarrow\Phi)\Leftrightarrow(\Uppsi\Rightarrow(\exists x)\Phi)$

\end{enumerate}

\end{lemma}

\subsection*{Логическая равносильность формул алгебры предикатов}

\begin{definition}
    Формулы алгебры предикатов 
$\Phi,\Uppsi$ называются \textit{логически равносильными}, если 
результат применения к ним логической 
операции эквивалентность $\Phi\Leftrightarrow\Uppsi$ является 
тавтологией. 
В этом случае записывают $\Phi\equiv\Uppsi$, или просто 
$\Phi = \Uppsi$.
Таким образом, $\Phi = \Uppsi$ означает, что $\models\Phi\Leftrightarrow\Uppsi$.
\end{definition}

\begin{theorem}[Теорема 1. Взаимосвязь между кванторами]
    Для любой формулы $\Phi$ справедливы равенства:

    \quad $(\forall x)(\forall y)\Phi = (\forall y)(\forall x)\Phi$, \quad
    $(\exists x)(\exists y)\Phi = (\exists y)(\exists x)\Phi$.

    С другой стороны, если в формулу $\Phi$ предметные переменные $x,y$ входят свободно, то равенство $(\forall y)(\exists x)\Phi = (\exists x)(\forall y)\Phi$ не выполняется, так как в этом случае формула $(\forall y)(\exists x)\Phi \Rightarrow (\exists x)(\forall y)\Phi$ не является тавтолгией
\end{theorem}

\begin{theorem}[Теорема 2]
    Пусть формула $\Phi(x)$ не содержит предметную переменную $y$ и формула $\Phi(y)$ получается из $\Phi(x)$ заменой всех свободных вхождений переменной $x$ на предметную переменную $y$.

    Тогда формулы $(\forall x)\Phi(x)$ и $(\exists x)\Phi(x)$ будут логически равносильны соответственно формулам $(\forall y)\Phi(y)$ и $(\exists y)\Phi(y)$, т.е. выполняются равенства:

    $(\forall x)\Phi(x) = (\forall y)\Phi(y)$ и $(\exists x)\Phi(x)=(\exists y)\Phi(y)$.
\end{theorem}

\begin{theorem}[Теорема 3. Законы де Моргана для кванторов] 
    Для любой формулы $\Phi$ справедливы следующие утверждения:

    $\lnot(\forall x)\Phi=(\exists x)\lnot\Phi$ \qquad 
    $\lnot(\exists x)\Phi=(\forall x)\lnot \Phi$

    $(\forall x)\Phi =\lnot(\exists x)\lnot\Phi$ \qquad
    $(\exists x)\Phi = \lnot(\forall x)\lnot\Phi$
\end{theorem}

\begin{theorem}[Теорема 4. Взаимосвязь кванторов с конъюнкцией и дизъюнкцией]
    Для любых формул $\Phi,\Uppsi$ справедливы следующие утверждения:

    $(\forall x)(\Phi\land\Uppsi) = (\forall x)\Phi\land(\forall x)\Uppsi$

    $(\exists x)(\Phi\lor\Uppsi) = (\exists x)\Phi\lor(\exists x)\Uppsi$
    
    Если в формулу $\Uppsi$ предметная переменная $x$ не входит свободно, то справедливы такэе утверждения:

    $(\forall x)\Phi\pi\Uppsi = (\forall x)(\Phi\pi\Uppsi)$,

    $(\exists x)\Phi\pi\Uppsi = (\exists x)(\Phi\pi\Uppsi)$,

    где $\pi$ -- символ одной из операций $\land,\lor$.
\end{theorem}

\begin{theorem}[Теорема 6. Взаимосвязь кванторов с импликацией]
    Если в формулу $\Phi$ предметная переменная $x$ не входит свободно, то для любой формулы $\Uppsi$ справедливы следующие утверждения:

    $(\forall x)(\Phi\Rightarrow\Uppsi) = \Phi \Rightarrow (\forall x)\Uppsi$

    $(\exists x)(\Phi\Rightarrow\Uppsi) = \Phi \Rightarrow (\exists x)\Uppsi$

    Если же предметная переменная $x$ не входит свободно в формулу $\Uppsi$, то для любой формулы $\Phi$ справедливы утверждения:

    $(\forall x)(\Phi\Rightarrow\Uppsi) = (\exists x)\Phi\Rightarrow\Uppsi$

    $(\exists x)(\Phi\Rightarrow\Uppsi) = (\forall x)\Phi\Rightarrow\Uppsi$
\end{theorem}

\begin{corollary}[Следствие 7]
    Любая формула $\Phi$ представляется в следующем виде:
    $$\Phi = (K_1 x_1)\ldots(K_n x_n)\Uppsi,$$
    
    где $K_1,\ldots,K_n$ -- некоторые кванторы и $\Uppsi$ -- формула без кванторов.

    Таким образом, каждая формула $\Phi$ логически равносильна формуле $(K_1 x_1)\ldots(K_n x_n)\Uppsi$, в которой все кванторы стоят в самом начале формулы и которая называется \textit{предваренной нормальной формой} (cокращенно ПНФ) формулы $\Phi$.

\end{corollary}