\subsection*{Формулы алгебры высказываний}
\begin{definition}
    Свойства алгебры всказываний $\mathscr{P}$ описываются с помощью формул, которые строятся из переменных символов с помощью знаков логических операций. Такие формулы принято называть также \textit{пропозициональными формулами}.
\end{definition}

\begin{definition}
    Символы логических операций $\lnot$, $\land$, $\lor$, $\Rightarrow$, $\Leftrightarrow$, которые называются \textit{пропозициональными связками}.
\end{definition}

\begin{definition}
    Переменные символы $X, Y, Z,\dots$, которые используются для обозначения высказываний, называются \textit{пропозициональными переменными}.
\end{definition}

\begin{definition}
    \textit{Формулы} алгебры высказываний индуктивно определяются по правилам:

    \begin{enumerate}
        \item Каждая пропозициональная переменная является формулой
        \item Если $\Phi$, $\Uppsi$  формулы, то формулами являются также выражения
        ($\lnot\Phi$), ($\Phi\land\Uppsi$), ($\Phi\lor\Uppsi$), ($\Phi\Rightarrow\Uppsi$), ($\Phi\Leftrightarrow\Uppsi$)
    \end{enumerate}

Множество всех формул алгебры высказываний обозначим $\mathcal{F}_{AB}$
\end{definition}

Для упрощения записи формул скобки в них по возможности опускаются с учетом следующего \textbf{приоритета выполнения операций:} $\lnot, \land, \lor$ и остальные.

Так, формула $((((\lnot X)\land (\lnot Y))\lor(\lnot(\lnot Z)))\Rightarrow(X \lor (\lnot Y)))$
сокращенно записывается в виде $\lnot X \land Y \lor \lnot\lnot Z \Rightarrow X \lor \lnot Y$.

Если в формулу $\Phi$ входят переменные $X_1, \dots, X_n$, то записывают $\Phi = \Phi(X_1,\dots,X_n)$.

Из индуктивного опеределения формул следует, что если в формулу $\Phi$ вместо переменных $X_1,\dots,X_n$ подставить произвольные конкретные высказывания $A_1,\dots,A_n$, то получится некоторое сложное высказывание $\Phi(A_1,\dots,A_n)$.

Истинностное значение высказывания $\lambda(\Phi(A_1,\dots,A_n))$ определяется истинностными значениями исходных высказываний $\lambda(A_1),\dots,\lambda(A_n)$ согласно таблицам истинностных значений логических операций $\lnot $, $\land $, $\lor $, $\Rightarrow $, $\Leftrightarrow $.

Формула $\Phi$ определяет функцию $n$ переменных $F_\Phi$, которая каждому упорядоченному набору $(\lambda(X_1),\dots,\lambda(X_n))$ $n$ элементов множества {0,1} ставит в соответствие элемент $\lambda(\Phi(X_1,\dots,X_n))$ этого же множества.

\begin{definition}
    Функция $F_\Phi$ называется \textit{истинностной функцией формулы $\Phi$} и графически представляется \textit{истинностной таблицей}.

    Такая таблица содержит $2^n$ строк и имеет ожно из возможных $2^{2^n}$ возможных распределений 0 и 1 в последнем столбце.
\end{definition}

\begin{example}
    Составим таблицу истинности для формулы 
    
    $(P\Rightarrow Q) \Leftrightarrow (\lnot Q \Rightarrow \lnot P)$

    \begin{center}
        \begin{tabular}{|c|c|c|c|c|c|c|}
            \hline
            $P$ & $Q$ & $P \Rightarrow Q$ & $\lnot Q$ & $\lnot P$ & $\lnot Q \Rightarrow \lnot P$ & $(P\Rightarrow Q)\Leftrightarrow (\lnot Q \Rightarrow \lnot P$) \\ \hline
            0   & 0   & 1                 & 1         & 1         & 1                             & 1                                                              \\ \hline
            0   & 1   & 1                 & 0         & 1         & 1                             & 1                                                              \\ \hline
            1   & 0   & 0                 & 1         & 0         & 0                             & 1                                                              \\ \hline
            1   & 1   & 0                 & 0         & 0         & 1                             & 1                                                              \\ \hline
            \end{tabular}
    \end{center}

    Таблица показывает, что какого бы истинностного значения высказывания ни подставлялось в данную формулу вместо пропозициональных переменных $P$ и $Q$, формула всегда превращается в истинностное высказывание.
\end{example}

\begin{definition}
    Формула $\Phi$ называется:
    \begin{itemize}
        \item \textit{Тавтологией} (или \textit{тождественно истинной формулой}) и обозначается $\vDash\Phi$, если ее истинностная функция тождественно равна 1
        \item \textit{Противоречием} (или \textit{тождественно ложной формулой}), если ее истинностная функция тождественно равна 0
        \item \textit{Выполнимой}, если ее истинностная функция не равна тождественно 0
        \item \textit{Опровержимой}, если ее истинностная функция не равна тождественно 1
    \end{itemize}
\end{definition}