%Тавтологии. Методы доказательства тавтологий
%Логическая равносильность формул. Равносильные преобразования формул
\subsection*{Тавтологии}
\begin{definition}
    Формула $\Phi$ называется \textit{тавтологией} (или \textit{тождественно истинной формулой}) и обозначается $\vDash\Phi$, если ее истинностная функция тождественно равна 1.
\end{definition}

Тавтологии являются общими схемами построения истинных высказываний и в этом смысле выражают некоторые \textit{логические законы}.

Примерами таких законов являются:

$\vDash X \lor \lnot X$ "--- закон исключенного третьего

$\vDash\lnot\lnot X \Leftrightarrow X$ "--- закон двойного отрицания

$\vDash\lnot(X \land \lnot X)$ "--- закон противоречия

$\vDash(X\Rightarrow Y) \Leftrightarrow (\lnot Y \Rightarrow \lnot X)$ "--- закон контрапозиции

\subsection*{Методы доказательства тавтологий}
Новые тавтологии можно получить с помощью следующего правила:
\begin{theorem}[Правило подстановки]
    Если $\vDash \Phi (X_1,\dots,X_n)$, то для любых формул $\Phi_1,\dots,\Phi_n$ тавтологией является формула $\Phi(\Phi_1,\dots,\Phi_n)$.
\end{theorem}

\textbf{Существуют различные методы доказательства тавтологий:}
\begin{enumerate}
    \item С помощью таблиц истинности
    \item Метод от противного
\end{enumerate}

\subsection*{Алгоритм проверки тождественной истинности формулы Ф:}
1.Рассмотреть формулу $\Uppsi=\lnot \Phi$ и найти ее КНФ $\Uppsi = D_1\land\ldots\land D_m$.

2. Найти резолютивный вывод значения 0 из множества $S = \{D_1,\ldots,D_m\}$.

3. Если такой вывод существует, то выполняется $\models\Phi$.
