\subsection*{Логическое следование формул алгебры предикатов}
С помощью логического следования формул 
определяются общие способы доказательства 
взаимосвязи между истинностными значениями 
утверждений посредством исследования 
формальной структуры этих утверждений.

\begin{definition}
    Формула $\Phi$ алгебры предикатов называется \textit{логическим следствием} формулы $\Uppsi$, если $\models\Uppsi\Rightarrow\Phi$, т.е. в любой интерпретации $M$ формула $\Phi$ выполняется при любой оценке предметных переменных $a$, при которой выполняется формула $\Uppsi$.
\end{definition}

\begin{definition}
    Формула $\Phi$ называется \textit{логическим следствие множества формул $\Gamma $}, если в любой интерпретации $M$ формула $\Phi $ выполняется при любой оценке предметных переменных $a$, при которой выполняются все формулы из $\Gamma$.

    Такое логическое следствие обозначается $\Gamma\models\Phi$ и называется \textit{логическим следованием}. При этом формулы из $\Gamma $ называются \textit{посылками} и формула $\Phi$ -- \textit{следствием} логического следования $\Gamma\models\Phi$.

    В случае, когда $\Gamma = {\Phi_1,\ldots,\Phi_m}$ записывают $\Phi_1,\ldots,\Phi_m\models\Phi$.
\end{definition}

\begin{definition}
    Множество формул $Gamma$ называется \textit{противоречивым}, если из него логическим следует любая (в том числе и тождественно ложная) формула $\Phi$. 

    Символически это записывается $\Gamma\models$.
\end{definition}

\begin{lemma}[Лемма 1. Критерии логического следования]
    Условие $\Phi_1,\ldots,\Phi_m\models\Phi$ равносильно каждому из следующих условий:
    \begin{itemize}
        \item $\Phi_1\land\ldots\land\Phi_m\models\Phi$
        \item $\models\Phi_1\land\ldots\land\Phi_m\Rightarrow\Phi$
        \item $\Phi_1,\ldots,\Phi_m,\lnot\Phi\models$
    \end{itemize}
    В частности, $\Phi\models\Uppsi$ равносильно $\models\Phi\Rightarrow\Uppsi$. 

    Отсюда также следует, что $\Phi=\Uppsi$ равносильно тому, что $\Phi\models\Uppsi$ и $\Uppsi\models\Phi$.
\end{lemma}

\subsection*{Основные правила логического следования}
\begin{enumerate}
    \item \textit{Правило отделения} (или правило \textit{модус поненс} -- от лат. modus ponens)
    $$\Phi,\Phi\Rightarrow\Uppsi\models\Uppsi$$
    \item \textit{Правило модус толенс} (от лат. modus tollens)
    $$\Phi\Rightarrow\Uppsi, \lnot\Uppsi\models\lnot\Phi$$
    \item \textit{Правило контрапозиции}
    $$\Phi\Rightarrow\Uppsi\models\lnot\Uppsi\Rightarrow\lnot\Phi$$
    \item \textit{Правило цепного заключения}
    $$\Phi_1\Rightarrow\Phi_2,\Phi_2\Rightarrow\Phi_3\models\Phi_1\Rightarrow\Phi_3$$
\end{enumerate}