\subsection*{Элиминация кванторов существования}
Пусть замкнутая формула исчисления предикатов $\Phi$ находится в ПНФ:
$$\Phi = (K_1 x_1)\ldots(K_n x_n)\Uppsi,$$
где $K_1,\ldots,K_n$ -- некоторые кванторы и $\Uppsi = \Uppsi(x_1,\ldots,x_n)$ -- конъюнктивное ядро формулы $\Phi$, т.е. бескванторная формула со свободными переменными $x_1,\ldots,x_n$, находящаяся в ПНФ.

В кванторной приставку формулы $\Phi$ можно удалить любой квантор существования $(\exists x_s)$ для $1\leq s \leq n$ по следующему правилу:
\begin{enumerate}
    \item Если левее квантора существования $\exists x_s$ в формуле $\Phi$ не стоит никакой квантор общности, то выбираем новый предметный символ $c$, заменяем этим символов $c$ все вхождения переменной $x_s$ в конъюнктивное ядро формулы $\Phi$ и вычеркиваем $(\exists x_s)$ из кванторной приставки формулы $\Phi$.

    \item Если же левее квантора существования $\exists x_s$ стоят кванторы общности
    $$(\forall x_{s_1}),\ldots,(\forall x_{s_m})$$
    для значений $1\leq s_1\leq\ldots\leq s_m\leq s$, то выбираем новый $m-$арный функциональный символ $f$, заменяем все вхождения переменной $x_s$ в конъюнктивное ядро формулы $\Phi$ выражением $f(x_{s_1},\ldots,x_{s_m})$ и вычеркиваем  $(\exists x_s)$ из кванторной приставки формулы $\Phi$.
\end{enumerate}

В результате такой замены всех кванторов существования в формуле $\Phi$ получим замкнутую ПНФ $\Phi'$, кванторная приставка которой получается из кванторной приставки формулы $\Phi$ удалением всех кванторов существования и которая содержит новые символы -- функциональные или предметные.

При этом формула $\Phi$ выполнима или противоречива одновременно с формулой $\Phi'$.

Рассмотренный прием удаления квантора существования был введен Скулемом и называется \textit{скулемизацией формул}. Вводимые в процессе скулемизации новые функциональные и предметные символы называются \textit{функторами Скулема} или \textit{скулемовскими функциями}.

Полученную в результате скулемизации замкнутую ПНФ $\Phi'$ называют \textit{скулемовской стандартной формой} (сокращенно ССФ).

\begin{theorem}
    Любая замкнутая формула исчисления предикатов $\Phi$ эффективно преобразуется (с помощью определенного алгоритма) в логически экваивалентную ей скулемовскую стандартную форму $\Phi'$, которая называется \textit{скулемовской стандартной формой} (сокращенно ССФ) формулы $\Phi$.

    При этом формула $\Phi$ выполнима или противоречива одновременно с ее ССФ.
\end{theorem}

\begin{example}
    Результатом скулемизации формулы
    $$(\forall x)(\exists z)(\forall y)(\exists w)((\lnot P(x)\lor R(y))\land P(z)\land \lnot R(w))$$ является следующая ССФ
    $$(\forall x)(\forall y)((\lnot P(x)\lor R(y))\land P(f(x))\land\lnot R(g(x,y))).$$
\end{example}