\begin{definition}
    Результатом действия квантора общности $(\forall x_1)$ по переменной $x_1$ на $n-$местный предикат $P(x_1,\ldots,x_n)$ называется \\ $(n-1)-$местный предикат $(\forall x_1)P(x_1,x_2,\ldots,x_n)$, который зависит от переменных $x_2,\ldots,x_n$ и который при значениях $x_2=a_2,\ldots,x_n=a_n$ в том и только том случае истинен на множестве $M$ допустимых значений переменной $x_1$, если при любых значениях $x_1=a_1 \in M$ высказывание $P(a_1,a_2,\ldots,a_n)$ истинно.
\end{definition}

\begin{definition}
    Резлуьтатом действия квантора существования $(\exists x_1)$ по переменной $x_1$ на $n-$местный предикат $P(x_1,\ldots,x_n)$ называется $(n-1)-$местный предикат $(\exists x_1)P(x_1,x_2,\ldots,x_n)$, который зависит от переменных $x_2,\ldots,x_n$ и который при значениях $x_2=a_2,\ldots,x_n=a_n$ в том и только том случае истинен на множестве $M$ допустимых значений переменной $x_1$, если при некотором значении $x_1=a_1 \in M$ высказывание $P(a_1,a_2,\ldots,a_n)$ истинно.
\end{definition}

\begin{definition}
    \textit{Квантор существования и единственности $(\exists ! x)$} определяется для сокращения записи следующей формулы \\
    $(\exists ! x)P(x)=(\exists x)(P(x)\land ((\forall y)(P(y)\Rightarrow x = y))).$

    Результат действия такого квантора на предикат $P(x)$ обозначается $(\exists ! x)P(x)$ и читается <<существует и единственен $x$, для которого выполняется $P(x)$>>.
\end{definition}

\begin{definition}
    \textit{Ограниченный квантор существования $\exists Q(x)$} определяется как сокращение записи следующей формулы $(\exists x)(Q(x)\land\ldots)$.

    Результат действия такого квантора на предикат $P(x)$ обозначается $(\exists Q(x))P(x)=(\exists x)(Q(x)\land P(x))$ и читается <<существует $x$, удовлетворяющий $Q(x)$, для которого выполняется $P(x)$>>.
\end{definition}

\begin{definition}
    \textit{Ограниченный квантор общности $\forall Q(x)$} определяется как сокращение записи следующей формулы $(\forall x)(Q(x)\Rightarrow \ldots)$.

    Результат действия такого квантора на предикат $P(x)$ обозначается $(\forall Q(x))P(x) = (\forall x)(Q(x)\Rightarrow P(x))$ и читается <<для всех $x$, удовлетворяющих $Q(x)$, выполняется $P(x)$>>.
\end{definition}

\begin{definition}
    Если в формулу $\Phi$ входят переменные $x_1,\ldots,x_n$, то записывают $\Phi(x_1,\ldots,x_n)$. Вхождение предметной переменной $x$ в формулу $\Phi$ называется \textit{связным}, если она находится в области действия одного из кванторов по этой переменной; в противном случае "--- \textit{свободным}.
\end{definition}
