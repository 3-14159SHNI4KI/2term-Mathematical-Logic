%Нормальные формы для формул алгебры высказываний
%Логическое следование формул. Методы доказательства логического следования формул
По определению формулы $\Phi$, $\Uppsi$ равносильны, значит их истинностные функции $F_\Phi$, $F_\Uppsi$ совпадают. 

Следовательно, отношение равносильности формул $\cong$ является отношением эквивалентности на множестве всех формул $\mathcal{F} _{AB}$, которое разбивает это множество на классы эквивалентности $[\Phi] = \{\Uppsi \in \mathcal{F} _{AB}: \Phi \cong \Uppsi\}$, определяемые формулами $\Phi\in\mathcal{F} _{AB} $.

Из основных равенств следует, что для каждой формулы $\Phi \in \mathcal{F} _{AB}$ можно указать равносильные ей формулы специального вида, содержащие только символы логических оераций $\lnot, \land, \lor$.

\begin{definition}
    \textit{Литерой} называется пропозициональная переменная $X$ или ее отрицание $\lnot X$. Обозначение: $X^\alpha$, где $\alpha\in\{0,1\}$.
    \begin{equation}
        X^\alpha
        \begin{cases}
            X^1 = X, & \text{если}~ \alpha = 1 \\
            X^0 = \lnot X, & \text{если}~ \alpha = 0
        \end{cases}
    \end{equation}
\end{definition}

\begin{definition}
    \textit{Конъюнктом} (соответственно, \textit{дизъюнктом}) называется литера или конъюнкция (соответственно, дизъюнкция) литер.

    Конъюнкт (дизъюнкт) называется \textit{совершенным}, если он содержит все пропозициональные переменные рассматриваемой формулы.
\end{definition}

\begin{definition}
    \textit{Конъюнктивной нормальной формой (КНФ)} называется дизъюнкт или конъюнкция дизъюнктов.

    \textit{Дизъюнктивной нормальной формой (ДНФ)} называется конъюнкт или дизъюнкция конъюнктов. 

    При этом КНФ (соответственно, ДНФ) называется \textit{совершенной}, если все ее дизъюнкты (соответственно, конъюнкты) содержат все пропозициональные переменные расматриваемой формулы.
\end{definition}

\begin{theorem}
    Любая формула равносильна некоторой ДНФ и некоторой КНФ.
\end{theorem}

\subsection*{Алгоритм приведения формулы Ф к ДНФ \\(соответсвенно, КНФ):}

\begin{enumerate}
    \item Выражаем все входящие в формулу Ф импликации и эквивалентности через конъюнкцию, дизъюнкцию и отрицание
    \item Согласно законам де Моргана все отрицания, стоящие перед скобками, вносим в эти скобки и сокращаем все двойные отрицания
    \item Согласно законам дистрибутивности преобразуем формулу так, чтобы все конъюнкции выполнялись раньше дизъюнкций (соответственно, чтобы все дизъюнкции выполнялись раньше конъюнкций)
\end{enumerate}

\begin{theorem}
    Любая выполнимая формула $\Phi = \Phi(X_1,\ldots,X_n)$ равносильна формуле вида
    $$\bigvee_{(a_1,\ldots,a_n)} (X_1^{\alpha_1}\land\ldots\land X_n^{\alpha_n}),$$
    где дизъюнкция берется по всем упорядоченным наборам 
    
    $(a_1,\ldots,a_n)\in\{0,1\}^n$, удовлетворяющим условию $\mathcal{F}_\Phi(\alpha_1,\ldots,\alpha_n) = 1$.
\end{theorem}

Такая формула определяется однозначно (с точностью до порядка членов конъюнкций и дизъюнкций) и называется \textit{совершенной дизъюнктивной нормальной формой (СДНФ)} формулы $\Phi$.

\begin{theorem}
    Любая опровержимая формула $\Phi = \Phi(X_1,\ldots,X_n)$ равносильна формуле вида
    $$\bigwedge_{(a_1,\ldots,a_n)} (X_1^{1-\alpha_1}\land\ldots\land X_n^{1-\alpha_n}),$$
    где конъюнкция берется по всем упорядоченным наборам

    $(a_1,\ldots,a_n)\in\{0,1\}^n$, удовлетворяющим условию $\mathcal{F}_\Phi(\alpha_1,\ldots,\alpha_n) = 0$.
\end{theorem}

Такая формула определяется однозначно (с точностью до порядка членов конъюнкций и дизъюнкций) и называется \textit{совершенной конъюнктивной нормальной формой (СКНФ)} формулы $\Phi$.

\subsection*{Алгоритм нахождения СДНФ и СКНФ формулы \\$\Phi = \Phi(X_1,\ldots,X_n)$:}
\begin{enumerate}
    \item Составить истинностную таблицу формулы $\Phi$ и добавить два столбца <<Совершенные конъюнкты>> и <<Совершенные дизъюнкты>>
    
    \item Если при значениях $\lambda(X_1) = k_1,\ldots,\lambda(X_n) = k_n$ значение $\lambda(\Phi(X_1,\ldots,X_n))$ формулы $\Phi$ равно 1, то в соответствующей строке таблицы в столбце <<Совершенные конъюнкты>> записываем конъюнкт $X_1^{k_1}\land\ldots\land X_n^{k_n}$ и в столбце <<Совершенные дизъюнкты>> делаем прочерк. При этом $X_i^1 = X_i$ и $X_i^0 = \lnot X_i$.
    
    \item Если при значениях $\lambda(X_1) = m_1,\ldots,\lambda(X_n) = m_n$ истинностное значение $\lambda(\Phi(X_1,\ldots,X_n))$ формулы $\Phi$ равно 0, то в соответствующей строке таблицы в столбце <<Совершенные дизъюнкты>> записываем дизъюнкт $X_1^{1-m_1}\lor\ldots\lor X_n^{1-m_n}$ и в столбце <<Совершенные конъюнкты>> делаем прочерк.
\end{enumerate}
\begin{center}
    \begin{tabular}{|c|c|c|c|c|c|c|}
        \hline
        $X_1$    & $\ldots$ & $X_n$    & $\ldots$ & $\Phi(X_1,\ldots,X_n)$ & \begin{tabular}[c]{@{}c@{}}Совершенные\\  конъюнкты\end{tabular} & \begin{tabular}[c]{@{}c@{}}Совершенные \\ дизъюнкты\end{tabular} \\ \hline
        $\ldots$ & $\ldots$ & $\ldots$ & $\ldots$ & $\ldots$               & $\ldots$                                                         & $\ldots$                                                         \\ \hline
        $k_1$    & $\ldots$ & $k_n$    & $\ldots$ & 1                      & $X_1^{k_1}\land\ldots\land X_n^{k_n}$                            & "---                                                             \\ \hline
        $\ldots$ & $\ldots$ & $\ldots$ & $\ldots$ & $\ldots$               & $\ldots$                                                         & $\ldots$                                                         \\ \hline
        $m_1$    & $\ldots$ & $m_n$    & $\ldots$ & 0                      & "---                                                             & $X_1^{1-m_1}\lor\ldots\lor X_n^{1-m_n}$                          \\ \hline
        $\ldots$ & $\ldots$ & $\ldots$ & $\ldots$ & $\ldots$               & $\ldots$                                                         & $\ldots$                                                         \\ \hline
        \end{tabular}
\end{center}

\begin{enumerate}
    \item[4.] СДНФ формулы $\Phi$ равна дизъюнкции полученных совершенных конъюнктов:
    $(X_1^{k_1}\land\ldots\land X_n^{k_n})\lor\ldots$
    \item[5. ] СКНФ формулы $\Phi$ равна конъюнкции полученных совершенных дизъюнктов: $(X_1^{1-m_1}\lor\ldots\lor X_n^{1-m_n})\land\ldots$
\end{enumerate}

\begin{example}
    Найдем СДНФ и СКНФ для формулы

    $\Phi(X,Y,Z) = \lnot(X\land Y)\Rightarrow\lnot(X\lor Z)$

    \begin{center}
        \begin{tabular}{|c|c|c|c|c|c|}
            \hline
            $X$ & $Y$ & $Z$ & $\Phi(X,Y,Z)$ & Совершенные конъюнкты     & Совершенные дизъюнкты              \\ \hline
            0   & 0   & 0   & 1             & $X^0\land Y^0 \land Z^0$  & "---                               \\ \hline
            0   & 0   & 1   & 0             & "---                      & $X^{1-0}\lor Y^{1-0} \lor Z^{1-1}$ \\ \hline
            0   & 1   & 0   & 1             & $X^0 \land Y^1 \land Z^0$ & "---                               \\ \hline
            0   & 1   & 1   & 0             & "---                      & $X^1 \lor Y^0 \lor Z^0$            \\ \hline
            1   & 0   & 0   & 0             & "---                      & $X^0 \lor Y^1 \lor Z^1$            \\ \hline
            1   & 0   & 1   & 0             & "---                      & $X^0 \lor Y^1 \lor Z^0$            \\ \hline
            1   & 1   & 0   & 1             & $X^1 \land Y^1 \land Z^0$ & "---                               \\ \hline
            1   & 1   & 1   & 1             & $X^1 \land Y^1 \land Z^1$ & "---                               \\ \hline
            \end{tabular}
    \end{center}

    СДНФ $\Phi = (\lnot X \land\lnot Y \lnot Z)\lor(\lnot X\land Y \lnot Z)\lor(X\land Y\land \lnot Z)\lor(X\land Y \land Z)$

    СКНФ $\Phi = (X\lor Y\lor \lnot Z)\land(X\lor \lnot Y \lor \lnot Z)\land(\lnot X \lor Y \lor Z)\land(\lnot X\lor Y \lor \lnot Z)$

\end{example}