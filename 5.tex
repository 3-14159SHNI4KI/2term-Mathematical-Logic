%Нормальные формы для формул алгебры высказываний
%Логическое следование формул. Методы доказательства логического следования формул
По определению формулы $\Phi$, $\Uppsi$ равносильны, значит их истинностные функции $F_\Phi$, $F_\Uppsi$ совпадают. 

Следовательно, отношение равносильности формул $\cong$ является отношением эквивалентности на множестве всех формул $\mathcal{F} _{AB}$, которое разбивает это множество на классы эквивалентности $[\Phi] = \{\Uppsi \in \mathcal{F} _{AB}: \Phi \cong \Uppsi\}$, определяемые формулами $\Phi\in\mathcal{F} _{AB} $.

Из основных равенств следует, что для каждой формулы $\Phi \in \mathcal{F} _{AB}$ можно указать равносильные ей формулы специального вида, содержащие только символы логических оераций $\lnot, \land, \lor$.

\begin{definition}
    \textit{Литерой} называется пропозициональная переменная $X$ или ее отрицание $\lnot X$. Обозначение: $X^\alpha$, где $\alpha\in\{0,1\}$.
    \begin{equation}
        X^\alpha
        \begin{cases}
            X^1 = X, & \text{если}~ \alpha = 1 \\
            X^0 = \lnot X, & \text{если}~ \alpha = 0
        \end{cases}
    \end{equation}
\end{definition}

\begin{definition}
    \textit{Конъюнктом} (соответственно, \textit{дизъюнктом}) называется литера или конъюнкция (соответственно, дизъюнкция) литер.

    Конъюнкт (дизъюнкт) называется \textit{совершенным}, если он содержит все пропозициональные переменные рассматриваемой формулы.
\end{definition}

\begin{definition}
    \textit{Конъюнктивной нормальной формой (КНФ)} называется дизъюнкт или конъюнкция дизъюнктов.

    \textit{Дизъюнктивной нормальной формой (ДНФ)} называется конъюнкт или дизъюнкция конъюнктов. 

    При этом КНФ (соответственно, ДНФ) называется \textit{совершенной}, если все ее дизъюнкты (соответственно, конъюнкты) содержат все пропозициональные переменные расматриваемой формулы.
\end{definition}

\begin{theorem}
    Любая формула равносильна некоторой ДНФ и некоторой КНФ.
\end{theorem}

\subsection*{Алгоритм приведения формулы Ф к ДНФ \\(соответсвенно, КНФ):}

\begin{enumerate}
    \item Выражаем все входящие в формулу Ф импликации и эквивалентности через конъюнкцию, дизъюнкцию и отрицание
    \item Согласно законам де Моргана все отрицания, стоящие перед скобками, вносим в эти скобки и сокращаем все двойные отрицания
    \item Согласно законам дистрибутивности преобразуем формулу так, чтобы все конъюнкции выполнялись раньше дизъюнкций (соответственно, чтобы все дизъюнкции выполнялись раньше конъюнкций)
\end{enumerate}

\begin{theorem}
    Любая выполнимая формула $\Phi = \Phi(X_1,\ldots,X_n)$ равносильна формуле вида
    $$\bigvee_{(a_1,\ldots,a_n)} (X_1^{\alpha_1}\land\ldots\land X_n^{\alpha_n}),$$
    где дизъюнкция берется по всем упорядоченным наборам 
    
    $(a_1,\ldots,a_n)\in\{0,1\}^n$, удовлетворяющим условию $\mathcal{F}_\Phi(\alpha_1,\ldots,\alpha_n) = 1$.
\end{theorem}

Такая формула определяется однозначно (с точностью до порядка членов конъюнкций и дизъюнкций) и называется \textit{совершенной дизъюнктивной нормальной формой (СДНФ)} формулы $\Phi$.

\begin{theorem}
    Любая опровержимая формула $\Phi = \Phi(X_1,\ldots,X_n)$ равносильна формуле вида
    $$\bigwedge_{(a_1,\ldots,a_n)} (X_1^{1-\alpha_1}\land\ldots\land X_n^{1-\alpha_n}),$$
    где конъюнкция берется по всем упорядоченным наборам

    $(a_1,\ldots,a_n)\in\{0,1\}^n$, удовлетворяющим условию $\mathcal{F}_\Phi(\alpha_1,\ldots,\alpha_n) = 0$.
\end{theorem}

Такая формула определяется однозначно (с точностью до порядка членов конъюнкций и дизъюнкций) и называется \textit{совершенной конъюнктивной нормальной формой (СКНФ)} формулы $\Phi$.

\subsection*{Алгоритм нахождения СДНФ и СКНФ формулы \\$\Phi = \Phi(X_1,\ldots,X_n)$:}
\begin{enumerate}
    \item Составить истинностную таблицу формулы $\Phi$ и добавить два столбца <<Совершенные конъюнкты>> и <<Совершенные дизъюнкты>>
    \item Если при значениях 
\end{enumerate}