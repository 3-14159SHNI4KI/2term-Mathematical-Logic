\begin{definition}
    Говорят, что язык $L$ принадлежит \textit{классу $\mathscr{P}$}, если он разрешим некоторой детерминированной машины Тьюринга $T$ с полиномиальной временной сложностью.
\end{definition}

\begin{definition}
    Распознавательная задача принадлежит классу $\mathscr{P}$, если ее язык принадлежит классу $\mathscr{P}$, т.е. эта задача решается с помощью полиномиального алгоритма -- некоторой детерминированной машины Тьюринга $T$ с полиномиальной временной сложностью.
\end{definition}

\begin{example}
    Задача вычисления НОД целых чисел принадлежит классу $\mathscr{P}$.
\end{example}

Помимо детерминированной машины Тьюринга $T = (\Sigma, Q, \delta, q_S, q_F)$ с одной программой $\delta$ в теории алгоритмов рассматриваются \textit{недетерминированные машины Тьюринга} $T = (\Sigma, Q, \delta_1, \delta_2, q_S, q_F)$ с двумя программами $\delta_1,\delta_2$, которая на каждом шаге вычислений случайным образом выбирает одну из этих двух программ и по ней выполняет изменение своей конфигурации.

\begin{definition}
    Язык $L$ принадлежит классу $\mathscr{NP}$, если он разрешим некоторой недетерминированной машины Тьюринга $T$ с полиномиальной временной сложностью.
\end{definition}

\begin{definition}
    Распознавательная задача принадлежит классу $\mathscr{NP}$, если ее язык принадлежит классу $\mathscr{NP}$, т.е. эта задача решается с помощью полиномиального недетерминированного алгоритма -- некоторой недетерминированной машины Тьюринга $T$ с полиномиальной временной сложностью.
\end{definition}

Это равносильно тому, что для объектов задачи $x$ имеется полиномиальный ограниченный эталон $y$, с помощью которого за полиномиальное время проверяется, что $x$ является или нет решением данной задачи.

\begin{example}
    Распознавательная задача ВЫП (SAT) выполнимаости формулы алгебры высказываний принадлежит классу $\mathscr{NP}$.
\end{example}

Очевидно, что $\mathscr{P}\subset\mathscr{NP}$, но вопрос о равенстве этих классов является важной \underline{открытой проблемой}.