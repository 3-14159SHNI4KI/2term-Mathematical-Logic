\subsection*{Аксиоматический подход}
Было определено множество формул алгебры высказываний $\mathscr{F}_{AB}$.

Затем было выделено подмножество этого множества $\mathscr{T}_{AB}\subset\mathscr{F}_{AB}$, состоящее из специальных формул -- тавтологий.

\begin{enumerate}
    \item \textit{Семантический подход}
    
    При этом в основе определения тавтологии лежит понятие интерпретации формул, т.е. придание некоторого конкретного содержательного смысла входящих в них переменных. Такой подход к логическим формулам носит теоретико"=множественный характер и называется \textit{семантическим}.

    \item \textit{Синтаксический подход}
    
    Альтернативой семантического подхода является \textit{синтаксический} подход, при котором логические формулы выводятся из первоначально выделенного множества формул -- аксиом по определенным правилам преобразования формул логического языка без привлечения вспомогательных теоретико"=множественных понятий.

    \item \textit{Аксиоматический подход}
    
    Построение математических теорий в виде аксиоматических теорий соответствующих формальных исчислений составляет суть \textit{аксиоматического метода} в математике.
\end{enumerate}

Простейшей аксиоматической теорией является \textit{аксиоматическая логика высказываний}, которая строится на основе соответствующего формального исчисления, называемого \textit{исчислением высказыываний (ИВ)}.

\subsection*{Исчисление высказываний}
Множество аксиом $Ax$(ИВ) исчисления выказываний описывается следующими тремя \textit{схемами аксиом}:

$(A_1)(\Phi\Rightarrow(\Uppsi\Rightarrow\Phi))$

$(A_2)((\Phi_1\Rightarrow(\Phi_2\Rightarrow\Phi_3))\Rightarrow((\Phi_1\Rightarrow\Phi_2)\Rightarrow(\Phi_1\Rightarrow\Phi_3)))$

$(A_3)((\lnot\Phi\Rightarrow\lnot\Uppsi)\Rightarrow((\lnot\Phi\Rightarrow\Uppsi)\Rightarrow\Phi))$

где $\Phi,\Uppsi,\Phi_i(i=1,2,3)$ -- произвольные формулы исчисления высказываний.

Исчисление высказываний имеет единственное \textit{правило вывода}, которое называется \textit{правилом заключения} или \textit{правилом modus ponens (MP)} и которое для произвольных формул исчисления высказыываний $\Phi,\Uppsi$ определяется по формуле $MP(\Phi\Rightarrow\Uppsi,\Phi)=\Uppsi$.

Символически это правило вывода записывается следующей схемой:
$$MP: \frac{\Phi\Rightarrow\Uppsi,\Phi}{\Uppsi}.$$

В основе алгоритма вывода \textit{теорем} исчисления высказываний лежит следующее понятие.

\begin{definition}
    Формула $\Phi$ называется \textit{теоремой исчисления высказываний}, если найдется такая конечная последовательность формул $\Phi_1,\ldots,\Phi_n$, в которой:

    \begin{enumerate}
        \item $\Phi_n = \Phi$
        \item Каждая формула $\Phi_i(1\leq i \leq n)$ либо является аксиомой, либо получается из некоторых двух предыдущих формул $\Phi_j,\Phi_k (1\leq j, k\leq i)$ по правилу вывода $MP$
    \end{enumerate}

    Последовательность формул $\Phi_1,\ldots,\Phi_n$ называется \textit{выводом} или \textit{доказательством} формулы $\Phi$.
\end{definition}

\begin{definition}
    Вывод формулы $\Phi$ сокращенно обозначают символом $\vdash\Phi$ и говорят, что <<$\Phi$ есть теорема>>.

    Множество всех таких теорем обозначается символом $Th$(ИВ) и называется \textit{теорией исчисления высказываний}.

    Главной целью построения исчисления высказываний является определение такой теории $Th$(ИВ), которая совпадает с множеством тавтологий $\mathscr{T}_{AB}$.
\end{definition}