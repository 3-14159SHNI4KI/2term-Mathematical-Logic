\subsection*{Проблема общезначимости формул алгебры предикатов}
Определение истинности формул вводится с помощью их интерпретаций в конкретных допустимых множествах $M$ с первоначально 
фиксированными предикатными символами этих формул. 

Так как множество таких интерпретаций бесконечно (они могут иметь как конечные, так и бесконечные области интерпретации), то в этом случае  проверить тождественную истинность рассматриваемой формулы на всех таких 
интерпретациях практически невозможно.

\begin{example}
    Формула 
    $$\Phi = (\forall x)\lnot P(x,x)\land(\forall x)(\forall y)(\forall z)(P(x,y)\land P(y,z)\Rightarrow P(x,z))\Rightarrow(\exists x)(\forall y)\lnot P(x,y)$$
    общезначима в любой конечной интерпретации, но не выполнима и интерпретации $M=N$ с отношением $P_M(x,y) = (x < y)$.
\end{example}

Альтернативный подход к проверке общезначимости 
формулы $\Phi$ основывается на попытке построения 
интерпретации, опровергающей данную формулу. 

Если из предположения существования такой 
интерпретации получается противоречие, то формула 
$\Phi$ общезначима. 

В противном случае на основе 
полученных условий для входящих в формулу $\Phi$
предикатов, алгебраических операций и констант 
строится интерпретация, опровергающая эту формулу 
$\Phi$, и в этом случае формула $\Phi$ не является 
общезначимой. 

\subsection*{Метод Эрбрана}

Таким образом, доказательство тождественной истинности замкнутой формулы $\Phi$ равносильно доказательству противоречивости ее отрицания $\lnot \Phi$.

Далее рассматривается задача доказательства противоречисовти замкнутой формулы $\Phi$.

\textbf{Правило 1.} Противоречивость замкнутой формулы алгебры предикатов $\Phi$ равносильна противоречивости ее скулемовской стандартной формы $\Phi'$, которая является универсально замкнутой формулой
$$\Phi'=(\forall_{i_1}x_{i_1})\ldots(\forall_{i_k}x_{i_k})\Uppsi$$
с конъюнктивным ядром $\Uppsi=D_1\land\ldots\land D_m$, где $D_1,\ldots,D_m$ -- некоторые дизъюнкты литер алгебры предикатов.

С другой стороны, универсально замкнутая формула $\Phi'$ противоречива в том и только том случае, если она невыполнима.

Доказательство противоречивости (т.е. невыполнимости) замкнутой формулы $\Phi'$ сводится к доказательству невыполнимости множества дизъюнктов 
$S=\{D_1,\ldots,D_m\}$.

Эрбран показал, что при доказательстве невыполнимости такого множества формул $S$ можно ограничиться рассмотрением интерпретаций в одной специальной области интерпретации, которая называется \textit{эрбрановским универсумом} и состоит из функциональных выражений от констант из $S$.

\textbf{Правило 2.} Доказательство противоречивости формул алгебры предикатов сводится к доказательству противоречивости конечных множеств дизъюнктов $S$.

Для этого строится резолютивный вывод 0 из множества дизъюнктов $S$.