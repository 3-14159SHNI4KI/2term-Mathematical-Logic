\subsection*{Проблема общезначимости формул алгебры предикатов}
Определение истинности формул вводится с помощью их интерпретаций в конкретных допустимых множествах $M$ с первоначально 
фиксированными предикатными символами этих формул. 

Так как множество таких интерпретаций бесконечно (они могут иметь как конечные, так и бесконечные области интерпретации), то в этом случае  проверить тождественную истинность рассматриваемой формулы на всех таких 
интерпретациях практически невозможно.

\begin{example}
    Формула 
    $$\Phi = (\forall x)\lnot P(x,x)\land(\forall x)(\forall y)(\forall z)(P(x,y)\land P(y,z)\Rightarrow P(x,z))\Rightarrow(\exists x)(\forall y)\lnot P(x,y)$$
    общезначима в любой конечной интерпретации, но не выполнима и интерпретации $M=N$ с отношением $P_M(x,y) = (x < y)$.
\end{example}

Альтернативный подход к проверке общезначимости 
формулы $\Phi$ основывается на попытке построения 
интерпретации, опровергающей данную формулу. 

Если из предположения существования такой 
интерпретации получается противоречие, то формула 
$\Phi$ общезначима. 

В противном случае на основе 
полученных условий для входящих в формулу $\Phi$
предикатов, алгебраических операций и констант 
строится интерпретация, опровергающая эту формулу 
$\Phi$, и в этом случае формула $\Phi$ не является 
общезначимой. 

\subsection*{Автоматическое доказательство теорем}

Существуют алгоритмы поиска доказательства, 
которые для общезначимых формул 
подтверждают, что эти формулы общезначимы, и 
для необщезначимых формул в общем случае не 
заканчивают свою работу.

Автоматические системы построения 
доказательств называют \textit{пруверами} и предъявляют 
им следующие требования:
\begin{enumerate}
    \item Корректность
    \item Полнота
    \item Эффективность
\end{enumerate}

Примером такого алгоритма является метод резолюций.

\subsection*{Метод резолюций в алгебре предикатов}

Первым шагом алгоритма Эрбрана является приведение рассматриваемой формулы к специальным нормальным формам, которые аналогичны ДНФ и КНФ для формул алгебры высказываний (сокращенно, АВ). 
