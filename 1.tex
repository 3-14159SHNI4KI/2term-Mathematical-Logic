\subsection*{Логика высказываний}
\begin{definition}
    \textit{Высказыванием} называется повествовательное предложение, о котором можно
    судить, истинное оно или ложное.

    Обозначаются высказывания: A, B, C...
\end{definition}

\begin{definition}
    \textit{Истинностное значение} высказывания А обозначается символом $\lambda(A)$ и определяется по формуле:

    $\lambda(A) = 1$, если высказывание А истинно

    $\lambda(A) = 0$, если высказывание А ложно
\end{definition}

\subsection*{Алгебра высказываний}
Из высказываний путем соединения их с помощью связок <<не>>, <<и>>, <<или>>, <<следует>>, <<равносильно>> можно составлять новые, более сложные высказывания.

При этом главное внимание уделяется функциональным зависимостям истинностных значений высказываний, в которых истинность или ложность новых высказываний определяется истинностью или ложностью составляющих их высказываний.

\begin{definition}
    \textit{Отрицанием высказывания А} называется высказывание $\lnot A$ (читается <<не $А$>>), которое истинно в том и только том случае, если высказывание $A$ ложно.
\end{definition}
Таблица истинностных значений операции отрицания:

\begin{center}
    \begin{tabular}{|c|c|}
    \hline
    $A$ & $\lnot A$ \\ \hline
    1   & 0         \\ \hline
    0   & 1         \\ \hline
    \end{tabular}
\end{center}

\begin{definition}
    \textit{Конъюнкцией высказываний A, B} называется высказывание $A\land B$ (читается <<$A$ и $B$>>), которое истинно в том и только том случае, если оба высказывания $A$, $B$ истинны.
\end{definition}

\begin{definition}
    \textit{Дизъюнкцией высказываний A, B} называется высказывание $A\lor B$ (читается <<$A$ или $B$>>), которое ложно в том и только том случае, если оба высказывания A, B ложны.
\end{definition}

\begin{definition}
    \textit{Импликацией высказываний A, B} называется высказывание $A\Rightarrow B$ (читается <<$A$ влечет $B$>>), которое ложно в том и только том случае, если высказывание $A$ истинно, а высказывание $B$ ложно.
\end{definition}

\begin{definition}
    \textit{Эквивалентностью высказываний A, B} называется высказывание $A\Leftrightarrow B$ (читается <<$A$ равносильно $B$>>), которое истинно в том и только том случае, если высказывания $A$ и $B$ имеют одинаковое истинностное значение.
\end{definition}

Таблица истинностных значений логических операций:
\begin{center}
    \begin{tabular}{|c|c|c|c|c|c|}
        \hline
        $A$ & $B$ & $A\land B$ & $A\lor B$ & $A\Rightarrow B$ & $A\Leftrightarrow B$ \\ \hline
        0   & 0   & 0          & 0         & 1                & 1                    \\ \hline
        0   & 1   & 0          & 1         & 1                & 0                    \\ \hline
        1   & 0   & 0          & 1         & 0                & 0                    \\ \hline
        1   & 1   & 1          & 1         & 1                & 1                    \\ \hline
        \end{tabular}
\end{center}

\begin{definition}
    \textit{Алгеброй высказываний} называется множество всех высказываний $\mathscr{P}$ с логическими операциями $\lnot $, $\land $, $\lor $, $\Rightarrow $, $\Leftrightarrow $.
\end{definition}