Под \textit{алгоритмом} интуитивно понимается совокупность инструкций, которые дают решение некоторой массовой задачи.

Общие свойства алгоритма:
\begin{enumerate}
    \item Дискретность алгоритма
    
    Расчлененность определяемого алгоритмом вычислительного процесса на отдельные этапы. При этом каждое из действий и весь алгоритм в целом обязательно завершаются.
    \item Детерминированность алгоритма

    Способ решения задачи определён однозначно в виде последовательности шагов.
    \item Элементарность шагов алгоритма
    \item Массовость алгоритма 

    Однажды составленный алгоритм должен подходить для решения подобных задач с разными исходными данными.
\end{enumerate}

Так как конструктивные объекты можно кодировать словами конечного алфавита $\varSigma$  (например, состоящего из двоичных символов 0 и 1), то алгоритм моделируется устройством, перерабатывающим слова алфавита $\varSigma$.

\textbf{Тезис Черча:}
Класс задач, решаемых в любой формальной модели алгоритма, совпадает с классом задач, которые могут быть решены интуитивно эффективными вычислениями, т.е. алгоритмическими методами.

\textbf{Модели алгоритма}

\textit{Алгоритмически неразрешимые} задачи привели к необходимости строго математического определения алгоритма.

Основные варианты математического определения алгоритма -- \textit{модели алгоритма}:

\begin{enumerate}
    \item \textit{Рекурсивная функция} (Клини, 1936 г.)
    \item \textit{Машина Тьюринга} (Пост и Тьюринг, 1936 г.)
    \item \textit{Нормальный алгорифм} (Марков, 1954 г.)
    \item \textit{Формальная грамматика} (Хомский, 1957 г.)
\end{enumerate}