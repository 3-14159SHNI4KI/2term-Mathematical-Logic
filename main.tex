\documentclass{Exams}
\usepackage{preamble}
\usepackage{fancyhdr}
\usepackage{mathrsfs}
\usepackage{upgreek}
\pagestyle{fancy}

%   Используйте для теоремы окружение 
%   \begin{theorem}[name] \end{theorem}
%   Аналогично:
%   определение (definition)
%   лемма (lemma),
%   следствие (corollary)
%   пример (example)
%   замечание (remark)
%   док-во (Proof) (именно с большой буквы)

% \Nat - множество натуральных чисел
% \Real - множество вещественных чисе
% \Zint - множество целых чисел
% \Complex - множество комплексных чисел
% \eps - эпсилон  
\fancyhead[RO]{\slshape А. Орехова \rightmark}
\fancyfoot[C]{\thepage}

\author{}
\title{Билеты к экзамену}
\date{2023}
\begin{document}

\subject{Математическая логика}
\ExamMakeTitle
\tableofcontents

\section{Логические операции над высказываниями}
\subsection*{Логика высказываний}
\begin{definition}
    \textit{Высказыванием} называется повествовательное предложение, о котором можно
    судить, истинное оно или ложное.

    Обозначаются высказывания: A, B, C...
\end{definition}

\begin{definition}
    \textit{Истинностное значение} высказывания А обозначается символом $\lambda(A)$ и определяется по формуле:

    $\lambda(A) = 1$, если высказывание А истинно

    $\lambda(A) = 0$, если высказывание А ложно
\end{definition}

\subsection*{Алгебра высказываний}
Из высказываний путем соединения их с помощью связок <<не>>, <<и>>, <<или>>, <<следует>>, <<равносильно>> можно составлять новые, более сложные высказывания.

При этом главное внимание уделяется функциональным зависимостям истинностных значений высказываний, в которых истинность или ложность новых высказываний определяется истинностью или ложностью составляющих их высказываний.

\begin{definition}
    \textit{Отрицанием высказывания А} называется высказывание $\lnot A$ (читается <<не $А$>>), которое истинно в том и только том случае, если высказывание $A$ ложно.
\end{definition}
Таблица истинностных значений операции отрицания:

\begin{center}
    \begin{tabular}{|c|c|}
    \hline
    $A$ & $\lnot A$ \\ \hline
    1   & 0         \\ \hline
    0   & 1         \\ \hline
    \end{tabular}
\end{center}

\begin{definition}
    \textit{Конъюнкцией высказываний A, B} называется высказывание $A\land B$ (читается <<$A$ и $B$>>), которое истинно в том и только том случае, если оба высказывания $A$, $B$ истинны.
\end{definition}

\begin{definition}
    \textit{Дизъюнкцией высказываний A, B} называется высказывание $A\lor B$ (читается <<$A$ или $B$>>), которое ложно в том и только том случае, если оба высказывания A, B ложны.
\end{definition}

\begin{definition}
    \textit{Импликацией высказываний A, B} называется высказывание $A\Rightarrow B$ (читается <<$A$ влечет $B$>>), которое ложно в том и только том случае, если высказывание $A$ истинно, а высказывание $B$ ложно.
\end{definition}

\begin{definition}
    \textit{Эквивалентностью высказываний A, B} называется высказывание $A\Leftrightarrow B$ (читается <<$A$ равносильно $B$>>), которое истинно в том и только том случае, если высказывания $A$ и $B$ имеют одинаковое истинностное значение.
\end{definition}

Таблица истинностных значений логических операций:
\begin{center}
    \begin{tabular}{|c|c|c|c|c|c|}
        \hline
        $A$ & $B$ & $A\land B$ & $A\lor B$ & $A\Rightarrow B$ & $A\Leftrightarrow B$ \\ \hline
        0   & 0   & 0          & 0         & 1                & 1                    \\ \hline
        0   & 1   & 0          & 1         & 1                & 0                    \\ \hline
        1   & 0   & 0          & 1         & 0                & 0                    \\ \hline
        1   & 1   & 1          & 1         & 1                & 1                    \\ \hline
        \end{tabular}
\end{center}

\begin{definition}
    \textit{Алгеброй высказываний} называется множество всех высказываний $\mathscr{P}$ с логическими операциями $\lnot $, $\land $, $\lor $, $\Rightarrow $, $\Leftrightarrow $.
\end{definition}

\section{Формулы и истинностные значения формул}
\subsection*{Формулы алгебры высказываний}
\begin{definition}
    Свойства алгебры всказываний $\mathscr{P}$ описываются с помощью формул, которые строятся из переменных символов с помощью знаков логических операций. Такие формулы принято называть также \textit{пропозициональными формулами}.
\end{definition}

\begin{definition}
    Символы логических операций $\lnot$, $\land$, $\lor$, $\Rightarrow$, $\Leftrightarrow$, которые называются \textit{пропозициональными связками}.
\end{definition}

\begin{definition}
    Переменные символы $X, Y, Z,\dots$, которые используются для обозначения высказываний, называются \textit{пропозициональными переменными}.
\end{definition}

\begin{definition}
    \textit{Формулы} алгебры высказываний индуктивно определяются по правилам:

    \begin{enumerate}
        \item Каждая пропозициональная переменная является формулой
        \item Если $\Phi$, $\Uppsi$  формулы, то формулами являются также выражения
        ($\lnot\Phi$), ($\Phi\land\Uppsi$), ($\Phi\lor\Uppsi$), ($\Phi\Rightarrow\Uppsi$), ($\Phi\Leftrightarrow\Uppsi$)
    \end{enumerate}

Множество всех формул алгебры высказываний обозначим $\mathcal{F}_{AB}$
\end{definition}

Для упрощения записи формул скобки в них по возможности опускаются с учетом следующего \textbf{приоритета выполнения операций:} $\lnot, \land, \lor$ и остальные.

Так, формула $((((\lnot X)\land (\lnot Y))\lor(\lnot(\lnot Z)))\Rightarrow(X \lor (\lnot Y)))$
сокращенно записывается в виде $\lnot X \land Y \lor \lnot\lnot Z \Rightarrow X \lor \lnot Y$.

Если в формулу $\Phi$ входят переменные $X_1, \dots, X_n$, то записывают $\Phi = \Phi(X_1,\dots,X_n)$.

Из индуктивного опеределения формул следует, что если в формулу $\Phi$ вместо переменных $X_1,\dots,X_n$ подставить произвольные конкретные высказывания $A_1,\dots,A_n$, то получится некоторое сложное высказывание $\Phi(A_1,\dots,A_n)$.

Истинностное значение высказывания $\lambda(\Phi(A_1,\dots,A_n))$ определяется истинностными значениями исходных высказываний $\lambda(A_1),\dots,\lambda(A_n)$ согласно таблицам истинностных значений логических операций $\lnot $, $\land $, $\lor $, $\Rightarrow $, $\Leftrightarrow $.

Формула $\Phi$ определяет функцию $n$ переменных $F_\Phi$, которая каждому упорядоченному набору $(\lambda(X_1),\dots,\lambda(X_n))$ $n$ элементов множества {0,1} ставит в соответствие элемент $\lambda(\Phi(X_1,\dots,X_n))$ этого же множества.

\begin{definition}
    Функция $F_\Phi$ называется \textit{истинностной функцией формулы $\Phi$} и графически представляется \textit{истинностной таблицей}.

    Такая таблица содержит $2^n$ строк и имеет ожно из возможных $2^{2^n}$ возможных распределений 0 и 1 в последнем столбце.
\end{definition}

\begin{example}
    Составим таблицу истинности для формулы 
    
    $(P\Rightarrow Q) \Leftrightarrow (\lnot Q \Rightarrow \lnot P)$

    \begin{center}
        \begin{tabular}{|c|c|c|c|c|c|c|}
            \hline
            $P$ & $Q$ & $P \Rightarrow Q$ & $\lnot Q$ & $\lnot P$ & $\lnot Q \Rightarrow \lnot P$ & $(P\Rightarrow Q)\Leftrightarrow (\lnot Q \Rightarrow \lnot P$) \\ \hline
            0   & 0   & 1                 & 1         & 1         & 1                             & 1                                                              \\ \hline
            0   & 1   & 1                 & 0         & 1         & 1                             & 1                                                              \\ \hline
            1   & 0   & 0                 & 1         & 0         & 0                             & 1                                                              \\ \hline
            1   & 1   & 0                 & 0         & 0         & 1                             & 1                                                              \\ \hline
            \end{tabular}
    \end{center}

    Таблица показывает, что какого бы истинностного значения высказывания ни подставлялось в данную формулу вместо пропозициональных переменных $P$ и $Q$, формула всегда превращается в истинностное высказывание.
\end{example}

\begin{definition}
    Формула $\Phi$ называется:
    \begin{itemize}
        \item \textit{Тавтологией} (или \textit{тождественно истинной формулой}) и обозначается $\vDash\Phi$, если ее истинностная функция тождественно равна 1
        \item \textit{Противоречием} (или \textit{тождественно ложной формулой}), если ее истинностная функция тождественно равна 0
        \item \textit{Выполнимой}, если ее истинностная функция не равна тождественно 0
        \item \textit{Опровержимой}, если ее истинностная функция не равна тождественно 1
    \end{itemize}
\end{definition}

\section{Тавтологии. Методы доказательства тавтологий}
%Тавтологии. Методы доказательства тавтологий
%Логическая равносильность формул. Равносильные преобразования формул
\subsection*{Тавтологии}
\begin{definition}
    Формула $\Phi$ называется \textit{тавтологией} (или \textit{тождественно истинной формулой}) и обозначается $\vDash\Phi$, если ее истинностная функция тождественно равна 1.
\end{definition}

Тавтологии являются общими схемами построения истинных высказываний и в этом смысле выражают некоторые \textit{логические законы}.

Примерами таких законов являются:

$\vDash X \lor \lnot X$ "--- закон исключенного третьего

$\vDash\lnot\lnot X \Leftrightarrow X$ "--- закон двойного отрицания

$\vDash\lnot(X \land \lnot X)$ "--- закон противоречия

$\vDash(X\Rightarrow Y) \Leftrightarrow (\lnot Y \Rightarrow \lnot X)$ "--- закон контрапозиции

\subsection*{Методы доказательства тавтологий}
Новые тавтологии можно получить с помощью следующего правила:
\begin{theorem}[Правило подстановки]
    Если $\vDash \Phi (X_1,\dots,X_n)$, то для любых формул $\Phi_1,\dots,\Phi_n$ тавтологией является формула $\Phi(\Phi_1,\dots,\Phi_n)$.
\end{theorem}

!Дописать методы проверки тавтологии

\subsection*{Алгоритм проверки тождественной истинности формулы Ф:}
1.Рассмотреть формулу $\Uppsi=\lnot \Phi$ и найти ее КНФ $\Uppsi = D_1\land\ldots\land D_m$.

2. Найти резолютивный вывод значения 0 из множества $S = \{D_1,\ldots,D_m\}$.

3. Если такой вывод существует, то выполняется $\models\Phi$.


\section{Логическая равносильность формул. Равносильные преобразования формул}
\subsection*{Логическая равносильность формул}
\begin{definition}
    Формулы $\Phi$, $\Uppsi$ называются \textit{логически равносильными} (или просто \textit{равносильными}), если они принимают одинаковые логические значения при любых истинностных значениях их переменных.

    Это равносильно условию $\vDash\Phi\Leftrightarrow\Uppsi$.
\end{definition}

\begin{definition}
    Для обозначения логически эквивалентных формул используется символическая запись $\Phi = \Uppsi$, или $\Phi \cong \Uppsi$.

    Такие выражения называют \textit{логическими равенствами} или просто \textit{равенствами} формул.
\end{definition}

\begin{lemma}[1]
    Справедливы следующие равенства формул:
    \begin{enumerate}
        \item Свойства ассоциативности дизъюнкции и конъюнкции:
        
        $X\lor(Y\lor Z) = (X\lor Y)\lor Z$

        $X\land (Y\land Z) = (X\land Y)\land Z$

        \item Свойства коммутативности дизъюнкции и конъюнкции:
        
        $X\lor Y = Y\lor X$

        $X\land Y = Y\land X$

        \item Свойства идемпотентности дизъюнкции и конъюнкции:
        
        $X\lor X = X$

        $X\land X = X$

        \item Законы дистрибутивности конъюнкции относительно дизъюнкции и дизъюнкции относительно конъюнкции:
        
        $X\land(Y\lor Z) = (X\land Y)\lor(X\land Z)$

        $X\lor(Y\land Z) = (X\lor Y)\land(X\lor Z)$

        \item Законы де Моргана:
        
        $\lnot(X\land Y) = \lnot X \lor \lnot Y$

        $\lnot(X\lor Y) = \lnot X \land \lnot Y$

        \item Законы поглощения:
        
        $(X\land Y)\lor X = X$

        $(X\lor Y)\land X = X$

        \item Закон двойного отрицания:
        
        $\lnot\lnot X = X$

        \item Взаимосвязь импликации с дизъюнкцией и конъюнкцией:
        
        $X\Rightarrow Y = \lnot X \lor Y$

        $X\Rightarrow Y = \lnot(X\land \lnot Y)$

        \item Взаимосвязь эквивалентности с импликацией, дизъюнкцией и конъюнкцией:
        
        $X\Leftrightarrow Y = (X\Rightarrow Y)\land(Y\Rightarrow X)$

        $X\Leftrightarrow Y = (\lnot X \lor Y) \land (X\lor \lnot Y)$
    \end{enumerate}
\end{lemma}

\subsection*{Равносильные преобразования формул}

\begin{lemma}[Правило замены] 
    Если формулы $\Phi$, $\Phi '$ равносильны, то для любой формулы $\Uppsi(X)$, содержащей переменную $X$, выполняется равенство: $\Uppsi(\Phi) = \Uppsi(\Phi')$.
\end{lemma}

Это правило означает, что при замене в любой формуле $\Uppsi = \Uppsi(\Phi)$ некоторой ее подформулы $\Phi$ на равносильную ей формулу $\Phi'$ получается формула $\Uppsi = \Uppsi(\Phi')$, равносильная исходной формуле $\Uppsi$.

Такие переходы называются \textit{равносильными преобразованиями формул}.

\begin{example}
    Формула $\Phi = (X\Rightarrow Y) \Rightarrow Z$ с помощью равенств 5, 7, 8 из леммы 1 равносильно преобразовывается следующим образом:

    $\Phi = (X\Rightarrow Y)\Rightarrow Z = \lnot(X\Rightarrow Y)\lor Z = \lnot(\lnot(X\land\lnot Y))\lor Z = (X\land \lnot Y)\lor Z$.
\end{example}


\section{Нормальные формы для формул алгебры высказываний}
%Нормальные формы для формул алгебры высказываний
%Логическое следование формул. Методы доказательства логического следования формул
По определению формулы $\Phi$, $\Uppsi$ равносильны, значит их истинностные функции $F_\Phi$, $F_\Uppsi$ совпадают. 

Следовательно, отношение равносильности формул $\cong$ является отношением эквивалентности на множестве всех формул $\mathcal{F} _{AB}$, которое разбивает это множество на классы эквивалентности $[\Phi] = \{\Uppsi \in \mathcal{F} _{AB}: \Phi \cong \Uppsi\}$, определяемые формулами $\Phi\in\mathcal{F} _{AB} $.

Из основных равенств следует, что для каждой формулы $\Phi \in \mathcal{F} _{AB}$ можно указать равносильные ей формулы специального вида, содержащие только символы логических оераций $\lnot, \land, \lor$.

\begin{definition}
    \textit{Литерой} называется пропозициональная переменная $X$ или ее отрицание $\lnot X$. Обозначение: $X^\alpha$, где $\alpha\in\{0,1\}$.
    \begin{equation}
        X^\alpha
        \begin{cases}
            X^1 = X, & \text{если}~ \alpha = 1 \\
            X^0 = \lnot X, & \text{если}~ \alpha = 0
        \end{cases}
    \end{equation}
\end{definition}

\begin{definition}
    \textit{Конъюнктом} (соответственно, \textit{дизъюнктом}) называется литера или конъюнкция (соответственно, дизъюнкция) литер.

    Конъюнкт (дизъюнкт) называется \textit{совершенным}, если он содержит все пропозициональные переменные рассматриваемой формулы.
\end{definition}

\begin{definition}
    \textit{Конъюнктивной нормальной формой (КНФ)} называется дизъюнкт или конъюнкция дизъюнктов.

    \textit{Дизъюнктивной нормальной формой (ДНФ)} называется конъюнкт или дизъюнкция конъюнктов. 

    При этом КНФ (соответственно, ДНФ) называется \textit{совершенной}, если все ее дизъюнкты (соответственно, конъюнкты) содержат все пропозициональные переменные расматриваемой формулы.
\end{definition}

\begin{theorem}
    Любая формула равносильна некоторой ДНФ и некоторой КНФ.
\end{theorem}

\subsection*{Алгоритм приведения формулы Ф к ДНФ \\(соответсвенно, КНФ):}

\begin{enumerate}
    \item Выражаем все входящие в формулу Ф импликации и эквивалентности через конъюнкцию, дизъюнкцию и отрицание
    \item Согласно законам де Моргана все отрицания, стоящие перед скобками, вносим в эти скобки и сокращаем все двойные отрицания
    \item Согласно законам дистрибутивности преобразуем формулу так, чтобы все конъюнкции выполнялись раньше дизъюнкций (соответственно, чтобы все дизъюнкции выполнялись раньше конъюнкций)
\end{enumerate}

\begin{theorem}
    Любая выполнимая формула $\Phi = \Phi(X_1,\ldots,X_n)$ равносильна формуле вида
    $$\bigvee_{(a_1,\ldots,a_n)} (X_1^{\alpha_1}\land\ldots\land X_n^{\alpha_n}),$$
    где дизъюнкция берется по всем упорядоченным наборам 
    
    $(a_1,\ldots,a_n)\in\{0,1\}^n$, удовлетворяющим условию $\mathcal{F}_\Phi(\alpha_1,\ldots,\alpha_n) = 1$.
\end{theorem}

Такая формула определяется однозначно (с точностью до порядка членов конъюнкций и дизъюнкций) и называется \textit{совершенной дизъюнктивной нормальной формой (СДНФ)} формулы $\Phi$.

\begin{theorem}
    Любая опровержимая формула $\Phi = \Phi(X_1,\ldots,X_n)$ равносильна формуле вида
    $$\bigwedge_{(a_1,\ldots,a_n)} (X_1^{1-\alpha_1}\land\ldots\land X_n^{1-\alpha_n}),$$
    где конъюнкция берется по всем упорядоченным наборам

    $(a_1,\ldots,a_n)\in\{0,1\}^n$, удовлетворяющим условию $\mathcal{F}_\Phi(\alpha_1,\ldots,\alpha_n) = 0$.
\end{theorem}

Такая формула определяется однозначно (с точностью до порядка членов конъюнкций и дизъюнкций) и называется \textit{совершенной конъюнктивной нормальной формой (СКНФ)} формулы $\Phi$.

\subsection*{Алгоритм нахождения СДНФ и СКНФ формулы \\$\Phi = \Phi(X_1,\ldots,X_n)$:}
\begin{enumerate}
    \item Составить истинностную таблицу формулы $\Phi$ и добавить два столбца <<Совершенные конъюнкты>> и <<Совершенные дизъюнкты>>
    \item Если при значениях 
\end{enumerate}

\section{Логическое следование формул. Методы доказательства логического следования формул}

\section{Метод резолюций в логике высказываний}

\section{Понятие предиката и его множества истинности. Перенесение на предикаты логических операций}

\section{Кванторы общности и существования, их действие на предикат. Свободные и связанные переменные}

\section{Формулы алгебры предикатов}

\section{Интерпретация формул алгебры предикатов}

\section{Логическое равенство формул алгебры предикатов. Свойства логических операций над предикатами}

\section{Логическое следование формул алгебры предикатов}

\section{Предваренная нормальная форма (ПНФ) формул алгебры предикатов}

\section{Скулемовская стандартная форма (ССФ) формул алгебры предикатов}

\section{Сведение проблемы общезначимости формул к проблеме противоречивости систем дизъюнктов}

\section{Унификация формул}

\section{Метод резолюций в логике предикатов}

\section{Аксиомы и правила вывода исчисления высказываний. Доказуемость формул}

\section{Теорема Геделя о полноте исчисления высказываний}

\section{Непротиворечивость и разрешимость исчисления высказываний}

\section{Аксиомы и правила вывода исчисления предикатов. Тождественная истинность выводимых формул}

\section{Полнота, непротиворечивость и неразрешимость исчисления предикатов}

\section{Понятие алгоритма и основные математические модели алгоритма}

\section{Разрешимые и полуразрешимые языки. Теорема Поста}

\section{Машины Тьюринга и вычисляемые ими функции}

\section{Распознаваемость языков машинами Тьюринга}

\section{Разрешимые, неразрешимые и распознавательные проблемы}

\section{Временная и ленточная сложность машины Тьюринга}

\section{Классы языков P и NP}

\section{Полиномиальные сведения проблем}

\section{NP"=трудные и NP"=полные языки}

\section{Основные NP"=полные проблемы}

\end{document}