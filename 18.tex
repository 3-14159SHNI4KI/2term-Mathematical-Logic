\subsection*{Автоматическое доказательство теорем}

Существуют алгоритмы поиска доказательства, 
которые для общезначимых формул 
подтверждают, что эти формулы общезначимы, и 
для необщезначимых формул в общем случае не 
заканчивают свою работу.

Автоматические системы построения 
доказательств называют \textit{пруверами} и предъявляют 
им следующие требования:
\begin{enumerate}
    \item Корректность
    \item Полнота
    \item Эффективность
\end{enumerate}

Примером такого алгоритма является метод резолюций.

\subsection*{Метод резолюций в исчислении предикатов}

Пусть $S$ -- множество дизъюнктов, $D_1,D_2$ -- дизъюнкты из $S$, которые \underline{не имеют общих переменных}, и $L_1,L_2$ -- литеры в $D_1$ и $D_2$ соответственно.

\begin{definition}
    Если множество формул $W=\{L_1,\lnot L_2\}$ имеет унификатор $\theta$, то дизъюнкт, получаемый из дизъюнкта $\theta(D_1)\lor\theta(D_2)$ вычеркиванием литер $\theta(L_1)$ и $\theta(L_2)$, называется \textit{бинарной резольвентой} дизъюнктов $D_1$ и $D_2$ и обозначается символом res$(D_1,D_2)$. При этом литеры $L_1$ и $L_2$ называются \textit{отрезаемыми литерами}.
\end{definition}

Если $\theta(D_1)=\theta(L_1)$ и $\theta(D_2)=\theta(L_2)$, то бинарную резольвенту дизъюнктов $D_1$ и $D_2$ полагаем равной 0.

Если дизъюнкты $D_1,D_2$ имеют общие переменные, то, заменив в формуле $D_2$ эти общие переменные на переменные, не встречающиеся в $D_1$ и $D_2$, получим дизъюнкт $D_2'$, который не имеет общих переменных с дизъюнктом $D_1$.
\begin{definition}
    \textit{Бинарной резольвентой} дизъюнктов $D_1$ и $D_2$ называется бинарная резольвента дизъюнктов $D_1$ и $D_2'$.
\end{definition}

\begin{example}
    Найдем бинарную резольвенту дизъюнктов $D_1=P_1(x)\lor P_2(x)$ и $D_2=\lor P_1(c)\lor P_3(x)$.

    Так как $D_1,D_2$ имеют общую переменную $x$, то заменим в формуле $D_2$ переменную $x$ на новую переменную $y$: $D_2'=\lnot P_1(c)\lor P_3(y)$.

    Выбираем в $D_1$ и $D_2'$ литеры $L_1=P_1(x)$ и $L_2=\lnot P_1(c)$ соответственно. Так как $\lnot L_2=L_2'=P_1(c)$ и формулы $L_1,L_2'$ имеют унификатор $\sigma = \begin{pmatrix} x \\ c \end{pmatrix}$, то бинарная резольвента формул $D_1$ и $D_2'$ получатеся из $\sigma(D_1)\lor\sigma(D_2')=P_1(c)\lor P_2(c)\lor\lnot P_1(c)\lor P_3(y)$ вычеркиванием литер $P_1(c)$ и $\lnot P_1(c)$.
\end{example}

\begin{definition}
    \textit{Резолютивный вывод} формулы $\Phi$ из множества дизъюнктов $S$ есть такая конечная последовательность дизъюнктов  $\Phi_1,\ldots,\Phi_k$, что:

    1. $\Phi_k=\Phi$

    2. Каждый дизъюнкт $\Phi_i$ или принадлежит множеству $S$, или является резольвентой некоторых дизъюнктов, предшествующих $\Phi_i$
\end{definition}

\begin{lemma}
    Резолютивный вывод из множества дизъюнктов $S$ сохраняет выполнимость формул.
\end{lemma}

\textbf{Правило 3.} (Основная теорема метода рехолюций)

Множество дизъюнктов $S$ противоречиво тогда и только тогда, когда существует резолютивный вывод нуля 0 из $S$.

\begin{example}
    Докажем противоречивость множества дизъюнктов $W=\{\Phi_1,\ldots,\Phi_6\}$, где 

    $\Phi_1 = P_1(a,f(b),f(c))$

    $\Phi_2=P_2(a)$

    $\Phi_3 = P_1(x,x,f(x))$

    $\Phi_4 = \lnot P_1(x,y,z)\lor P_3(x,z)$

    $\Phi_5 = \lnot P_2(x)\lor \lnot P_1(y,z,u)\lor\lnot P_3(x,u)\lor P_3(x,y)\lor P_3(x,z)$

    $\Phi_6 = \lnot P_3(a,c)$

    $\Phi_7 =$ res$(\Phi_2,\Phi_5) =$ res $(\Phi_2,\begin{pmatrix}
        x & y \\ a & z \end{pmatrix} (\Phi_5)) = \lnot P_1(z,z,u)\lor\lnot P_3(a,u)\lor P_3(a,z)$
    
    $\Phi_8 = $ res$(\Phi_3,\Phi_7) =$ res$(\Phi_3,\begin{pmatrix}
        z & u \\ x & f(x)
    \end{pmatrix}(\Phi_7)) = \lnot P_3(a,f(x))\lor P_3(a,x)$

    $\Phi_9 = $ res$\Phi_6,\Phi_8 = $ res$(\Phi_6, \begin{pmatrix}
        x \\ c
    \end{pmatrix}(\Phi_8)) = \lnot P_3(a,f(c))$

    $\Phi_{10} =$ res$(\Phi_4,\Phi_9) =$ res$(\begin{pmatrix}
        x & z \\ a & f(c)
    \end{pmatrix}(\Phi_4),\Phi_9) = \lnot P_1(a,y,f(c))$

    $\Phi_{11} =$ res$(\Phi_1,\Phi_{10}) =$ res$(\Phi_1,\begin{pmatrix}
        y \\ f(b)
    \end{pmatrix}(\Phi_{10})) = 0$
\end{example}

\subsection*{Применения метода резолюций исчисления предикатов}
Следующие задачи равносильны:
\begin{enumerate}
    \item Проверка тождественной истинности формул
    \item Проверка логического следования формул
    \item Проверка тождественной ложности формул
    \item Проверка противоречивости множества формул
    \item Проверка противоречивости множества дизъюнктов
\end{enumerate}

\begin{example}
    Методом резолюций доказать общезначимость формулы
    $$\Phi = ((\exists x)P(x)\Rightarrow(\forall x)R(x))\Rightarrow (\forall x)(P(x)\Rightarrow R(x)).$$

    Условие $\models\Phi$ равносильно $\lnot\Phi\models$.

    Для формулы $\Uppsi = \lnot\Phi$ найдем ПНФ и ССФ.

    ПНФ формулы $\Uppsi$: \quad
    $(\forall x)(\forall y)(\exists z)((\lnot P(x)\lor R(y))\land P(z)\lnot R(z))$

    ССФ формулы $\Uppsi$: \quad
    $(\forall x)(\forall y)((\lnot P(x)\lor R(y))\land P(f(x,y))\land\lnot R(f(x,y)))$

    Для доказательства невыполнимости этой формулы достаточно доказать противоречивость множества дизъюнктов ее конъюнктивного ядра

    $$S = \{\lnot P(x)\lor R(y), P(f(x,y)), \lnot R(f(x,y))\}$$

    Резолютивный вывод формулы 0 из множества дизъюнктов $S$:

    $\Phi_1 = \lnot P(x)\lor R(y)$

    $\Phi_2 = P(f(x,y))$

    $\Phi_3 = $ res$(\Phi_1.\Phi_2) =$ res$(\lnot P(x)\lor R(y), P(f(x,y)))=$ res$(\lnot P(x)\lor R(y), P(f(x_1,y_1))) = $ res$(\theta(\lnot P(x)\lor R(y)), P(f(x_1,y_1))) =$ res$(\lnot P(f(x_1,y_1))\lor R(y), P(f(x_1,y_1))) = R(y)$, где $\theta = \begin{pmatrix}
        x \\ f(x_1,y_1)
    \end{pmatrix}$

    $\Phi_4 = \lnot R(f(x,y))$

    $\Phi_5 =$ res$(\Phi_3,\Phi_4) =$ res$(R(y),\lnot R(f(x,y))) =$ res$(R(y),\lnot R(f(x_1,y_1))) = $ res$(\theta(R(y)),\lnot R(f(x_1,y_1))) = 0$, где $\theta = \begin{pmatrix}
        y \\ f(x_1,y_1)
    \end{pmatrix}$
\end{example}

\subsection*{Резолютивный вывод как средство вычисления}
Метод резолюций используется для решения следующей задачи:

Будет ли верно утверждение $\psi$, если известно, что верны утверждения $\Gamma = \{\varphi_1,\varphi_2,\ldots,\varphi_n\}$?

Здесь база знаний $\Gamma = \{\varphi_1,\varphi_2,\ldots,\varphi_n\}$.

Предложение $\psi$ -- запрос к базе знаний.

\textbf{Задача (неформальная)}: выяснить, является ли предложение $\psi$ следствием утверждений базы знаний $\Gamma$.

\textbf{Задача (неформальная)}: проверить, что $\psi$ выводится из $\Gamma$ по законам формальной логики.