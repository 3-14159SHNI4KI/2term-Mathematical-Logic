%\section{Понятие предиката и его множества истинности. Перенесение на предикаты логических операций}

%\section{Кванторы общности и существования, их действие на предикат. Свободные и связанные переменные}

%\section{Формулы алгебры предикатов}

%\section{Интерпретация формул алгебры предикатов}

\subsection*{Логика предикатов. Понятие предиката.}
Выразительные средства алгебры высказываний недостаточны для описания утверждений со сложной логической структурой субъектно-предикатных рассуждений, в которых используются не только понятие \textit{субъекта} (как объекта, о котором говорится в рассуждении), но и понятие \textit{предиката} (как выраженного сказуемыми свойства объектов рассуждения).

\begin{definition}
    \textit{Предикатом} называется утверждение, содержащее переменные $x_1,\ldots,x_n$, которое превращается в высказывание при замене этих переменных конкретными объектами из некоторой области возможных значений.

    Обозначаются предикаты $P,Q,\ldots$
\end{definition}

\begin{definition}
    Переменные $x_1,\ldots,x_n$ называются \textit{предметными} или \textit{индивидуальными переменными}. Число предметных переменных в предикате называется его \textit{арностью} или \textit{местностью}.

    Более точно, предикат $P$ с $n$ предметными переменными называется \textit{n-арным} или \textit{n-местным предикатом} и обозначается $P(x_1,\ldots,x_n)$.
\end{definition}

Предикат $P(x_1,\ldots,x_n)$ является функцией, которая каждому набору значений $x_1=a_1,\ldots,x_n=a_n$ его $n$ предметных переменных $x_1,\ldots,x_n$ ставит в соответствие некоторое высказывание $P(a_1,\ldots,a_n)$, имеющее определенное истинностное значение $\lambda(P(a_1,\ldots,a_n))$.

Если отвлечься от содержания высказываний и учитывать только их истинностные значения, то предикат можно рассматривать как функцию со значениями в множестве $\{0,1\}$.

Рассматривая такую функцию на некотором фиксированном множестве $M$ допустимых значений предметных переменных предиката, получим $n-$арное отношение на множестве $M$, состоящее из всех таких упорядоченных наборов $(a_1,\ldots,a_n)$ $n$ элементов $a_1,\ldots,a_n\in M$, для которых $P(a_1,\ldots,a_n)$ является истинным высказыванием.

Такое $n-$арное отношение обозначается символом $P^+$ и называется \textit{множеством истинности} предиката $P$ на множестве $M$. \\ \\ \\

Функция $P:M^n\rightarrow \{0,1\}$ определяется двумя множествами:

\begin{enumerate}
    \item \textit{Множество истинности: } 
    $$P^+=\{(a_1,\ldots,a_n)\in M^n:\lambda(P(a_1,\ldots,a_n))=1\}$$
    \item \textit{Множество ложности:} $$P^-=\{(a_1,\ldots,a_n)\in M^n:\lambda(P(a_1,\ldots,a_n))=0\}$$
\end{enumerate}

\begin{example}
    Пусть $M - $ множество студентов вуза.

    Предикаты:

    $P(x) - $ <<x есть студент 1-ой группы>>,

    $Q(x) - $ <<студент x есть отличник>>.

    Множеством истинности $P^+$ на множестве $M$ является множество студентов 1-ой группы вуза и множеством истинности $Q^+$ на множестве $M$ является множество всех отличников вуза.
\end{example}

\begin{example}
    Пусть $M - $ множество вещественных чисел $\mathbb{R}$.

    Предикаты:

    $P(x) - $ утверждение <<x>0>>,

    $Q(x) - $ утверждение <<$(x-1)(x^2-2) = 0$>>.

    Множеством истинности предиката $P$ на множестве $M = \mathbb{R}$ является множество всех положительных вещественных чисел и множеством истинности предиката $Q$ на множестве $M = \mathbb{R} $ является множество \\
    $Q^+ = \{1,\sqrt{2},-\sqrt{2}\}$.
\end{example}

\begin{definition}
    Предикат $P(x_1,\ldots,x_n)$ на множестве $M$ называется:

    \begin{itemize}
        \item \textit{Тождественно истинным}, если для любых значений $x_1=a_1\in M,\ldots,x_n=a_n \in M$ высказывание $P(a_1,\ldots,a_n)$ истинно, т.е. $P^+=M^n$.
        \item \textit{Тождественно ложным}, если для любых значений $x_1=a_1 \in M,\ldots,x_n=a_n \in M$ высказывание $P(a_1,\ldots,a_n)$ ложно, т.е. $P^+=\varnothing$.
        \item \textit{Выполнимым}, если для некоторых значений $x_1=a_1 \in M,\ldots,x_n=a_n \in M$ высказывание $P(a_1,\ldots,a_n)$ истинно, т.е. $P^+\neq \varnothing$.
        \item \textit{Опровержимым}, если для некоторых значений $x_1=a_1 \in M,\ldots,x_n=a_n \in M$ высказывание $P(a_1,\ldots,a_n)$ ложно, т.е. $P^+\neq M^n$.
    \end{itemize}
\end{definition}

\subsection*{Алгебра предикатов}
\begin{definition}
    \textit{Отрицание $n-$местного предиката $P(x_1,\ldots,x_n)$} определяется как $n-$местный предикат $\lnot P$, который при подстановке значений $x_1=a_1,\ldots,x_n=a_n$ превращается в высказывание $\lnot P(a_1,\ldots,a_n)$, являющееся отрицанием высказывания ${P(a_1,\ldots,a_n)}$.
\end{definition}

\begin{definition}
    \textit{Конъюнкция $n-$местных предикатов $P(x_1,\ldots,x_n)$ и $Q(x_1,\ldots,x_n)$} определяется как $n-$местный предикат $P\land Q$, который при подстановке значений $x_1=a_1,\ldots,x_n=a_n$ превращается в высказывание $P\land Q(a_1,\ldots,a_n)$, являющееся конъюнкцией высказываний $P(a_1,\ldots,a_n)$ и $Q(a_1,\ldots,a_n)$.
\end{definition}

Для любого множества $M$ допустимых значений предметных переменных предикатов множества истинности предикатов взаимосвязаны с логическими операциями по следующим формулам:

$(\lnot P)^+=M^n \backslash P+$

$(P\land Q)^+ = P^+\cap Q^+ $

$(P\lor Q)^+=P^+\cup Q^+$

$(P\Rightarrow Q)^+=(\lnot P)^+\cup Q^+$

$(P\Leftrightarrow Q)^+ = (P\Rightarrow Q)^+\cap(Q\Rightarrow P)^+$

\begin{example}
    Пусть на множестве вещественных чисел $\mathbb{R} $ предикат $P(x)$ выражается неравенством $f(x)\leq 0,$ и предикат $Q(x)$ выражается неравенством $g(x)\leq 0$.

    Тогда система неравенств
         $\begin{cases} f(x)\leq 0 \\ g(x)\leq 0 \end{cases}$
    определяется как конъюнкция предикатов $P\land Q$ и, значит, имеет множество решений $(P\cap Q)^+=P^+\cap Q^+$, равное пересечению множеств решений неравенств системы.

    Совокупность неравенств 
    $\left[ 
      \begin{gathered} 
        f(x)\leq 0, \\ 
        g(x)\leq 0 \\ 
      \end{gathered} 
\right.$ определяется как дизъюнкция предикатов $P\lor Q$ и, значит, имеет множество решений $(P\cup Q)^+=P^+\cup Q^+$, равное объединению множеств решений неравенств системы.
\end{example}
