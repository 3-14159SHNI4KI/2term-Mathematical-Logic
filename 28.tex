Определение распознавательной задачи:
\begin{definition}
    Важные математические проблемы имеют следующий вид.

    Имеется множество объектов $X$ и определенное подмножество $P\subset X$, требуется найти эффективную процедуру (т.е. алгоритм), с помощью которой для любого $x \in X$ можно определить за конечное число шагов, будет этот элемент обладать некоторым данным свойством $P$ или нет (т.е. $x\in P^+$ или $x\notin P^+$).

    \textit{Решением} такой проблемы является построение и обоснование искомого алгоритма.
\end{definition}

\textit{Массовые задачи} -- задачи распознавания и оптимизации.

\textbf{Пример массовых задач:}

\begin{itemize}
    \item СУМ -- задача сложения целых чисел
    \item ДЕЛ -- задача делимости целых чисел
    \item НОД -- задача нахождения наибольшего общего делителя двух целых чисел
    \item ВЫП (SAT) -- задача выполнимости формулы алгебры высказываний
    \item ТЕОРЕМА (THM) -- задача доказуемости формулы алгебры предикатов
\end{itemize}

Классификация распознавательных задач определяется классификацией кодирующих эти задачи языков.

\begin{definition}
    Распознавательная задача называется \textit{разрешимой} (соответственно, \textit{полуразрешимой}), если разрешим (соответственно, полуразрешим) кодирующий эту задачу язык.
\end{definition}

\begin{example}
    Распознавательная задача ВЫП (SAT) выполнимости формулы алгебры высказываний разрешима (с помощью алгоритма составления истинностных таблиц).
\end{example}

\begin{example}
    Распознавательная задача ТЕОРЕМА (THM) доказуемости формулы алгебры предикатов полуразрешима (с помощью понятия вывода формул), но не разрешима.
\end{example}