Доказательство $\mathscr{T}_{\text{АП}}\subset Th$(ИП) было получено австрийским математиком Куртом Гёделем в 1930 году.
\subsection*{Теорема полноты ИП}
Формула исчисления предикатов в тои и только том случае является тавтологией, если она есть теорема ИП, т.е. выполняется равенство $\mathscr{T}_{\text{АП}} = Th$(ИП).

Таким образом, ИП является адекватным инструментом получения логических законов.

\subsection*{Теорема о непротиворечивости ИП}
В исчислении предикатов невозможно доказать никакую формулу $\Phi$ вместе с ее отрицанием $\lnot\Phi$.

\subsection*{Теорема о неразрешимости ИП}
С другой стороны, английский математик Алонзо Чёрч в 1936 году доказал следующий принципиально важный результат.

Не существует универсальной эффективной процедуры (алгоритма), которая для любой формулы определяет, является ли эта формула теоремой ИП.