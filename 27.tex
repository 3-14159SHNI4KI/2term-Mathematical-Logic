Работа машины Тьюринга $T$ происходит под действием ее команд и заключается в изменении ее \textit{конфигураций}, описывающих состояния ленты и управляющего устройства, а также положение головки относительно ячеек ленты:

если лента находится в состоянии, которое описывается словом $\alpha a\beta$ (альфа а бета) над алфавитом $\Sigma$, и головка в состоянии $q$ просматривает на ленте ячейку с состоянием $\alpha$, то соответствующая конфигурация $K$ машины $T$ описывается выражением $M=\alpha qa\beta$, которое называется \textit{машинным словом}.

При этом $K$ называется \textit{начальной конфигурацией}, если описывающее ее машинное слово содержит символ начального состояния $q_S$, и \textit{заключительной конфигурацией}, если описывающее ее машинное слово содержит символ заключительного состояния $q_F$.

Программа указывает, что машина делает в каждый момент времени в зависимости от ее настоящей конфигурации $K$:

если $K$ -- заключительная конфигурация, то машина заканчивает работу, если же $K$ не является заключительной конфигурацией и описывается машинным словом $M=\alpha qa \beta$, то в программе $\Pi$ машина находит команду $T(q,a)$ с левой частью $qa$ и в зависимости от вида правой части такой машины $T(q,a)$ машина заменяет в просматриваемой ячейке букву $a$ на букву $a'$, состояние $q$ на состояние $q'$ и в зависимости от значения $X\in\{R,L,S\}$ сдвигает просматривающую головку либо в соседнюю правую ячейку при $X=R$, либо в соседнюю левую ячейку при $X=L$, либо оставляет головку на месте при $X=S$.

Изменение конфигураций $K_0,K_1,K_2,\ldots$ машины $T$ под действием команд происходит в дискретные моменты времени $t=0,1,2,\ldots$ и описывается преобразованием соответствующих машинных слов $M_0,M_1,M_2,\ldots$ по следующему правилу.

За один шаг работы машины $T$ ее машинное слово $M=\alpha qa \beta$ под действием команды $T(q,a)$ преобразуется в новое машинное слово $M'$ по формулам:

\begin{itemize}
    \item если $T(q,a)\rightarrow q'a'S,$ то $M'=\alpha q'a'\beta$
    \item если $T(q,a)\rightarrow q'a'R$ и $M=\alpha qab \beta'$, то $M'=\alpha a'q'b\beta'$
    \item если $T(q,a)\rightarrow q'a'R$ и $M=\alpha qa$, то $M'=\alpha a'q'*$
    \item если  $T(q,a)\rightarrow q'a'L$ и $M = \alpha' bqa \beta$, то $M'=\alpha' q'ba'\beta$
    \item если $T(q,a)\rightarrow q'a'L,$  и $M=qa\beta$, то $M'=q'*a'\beta$
\end{itemize}

Символически такое одношаговое преобразование машинных символов обозначается $M\rightarrow^T M'$.

Если существует такая последовательность преобразований машинных слов  $M_i\rightarrow^T M_{i+1}$ (где $i=0,1,\ldots,k-1$), для которой $M_0=M$ и $M_k=M'$, то пишут $M\Rightarrow^T M'$ и говорят, что \textit{машинное слово $M'$ получается из машинного слова $M$ с помощью машины $T$}.

\underline{Вход (начало работы) машины:}

слово $w\in\Sigma^*$ на ленте машины $T$ в начальном состоянии $q_S$.

\underline{Выход(завершение работы) машины:}

слово $w'\in\Sigma^*$ на ленте машины $T$ в заключительном состоянии $q_F$.

В этом случае говорят, что машина $T$ \textit{принимает} слово $w$ и выдает значение $w'=T(w)$. В результате машина $T$ определяет язык $L(T)\subset\Sigma^*$, который состоит из всех принимаемых машиной $T$ слов.

\begin{definition}
    Язык $L\subset\Sigma^*$ \textit{принимается} машиной Тьюринга, если $L=L(T)$ для некоторой машины Тьюринга $T$.
\end{definition}

Таким образом, любая машина Тьюринга $T$ определяет частичную функцию $f$ из $\Sigma^*$ в $\Sigma^*$, область определения которой $D_f$ состоит из всех слов алфавит $\Sigma$, которые принимает машина $T$, и значения которой для всех слов $w\in D_f$ определяются по формуле $f(w)=T(w)$.

\begin{example}
    Пусть машина Тьюринга $T$ имеет внешний алфавит $\Sigma=\{0,1\}$, внутренний алфавит $Q=\{q_S,q_F,q\}$ и программу $\Pi$, которая состоит из команд: 

    $q_S1\rightarrow q1R$, $q1\rightarrow q1R$, $q*\rightarrow q_F 1S$.

    Тогда слово $\alpha=11$ машиной $T$ перерабатывается в слово $\beta =111$, так как
    $$q_S11\rightarrow^T 1q1\rightarrow^T 11q*\rightarrow^T 11q_F1 \text{ и } T(11)=111.$$

    Легко видеть, что любое слово $\alpha=1^n$ над алфавитом $\Sigma = \{0,1\}$ машиной $T$ перерабатывается в слово $\beta=\alpha 1=1^{n+1}$.

    Это означает, что машина $T$ к любому слову над алфавитом $A=\{1\}$ приписывает справа символ 1.
\end{example}