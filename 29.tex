\subsection*{Сложность вычислений}
В качестве модели алгоритма рассматривается машина Тьюринга $T$, вычисляющая числовую функцию $f(x)$.

\begin{definition}
    \textit{Временной сложностью} машины $T$ называется функция $t_T(x)$, значение которой равно числу шагов работы машины $T$, сделанных при вычислении значения $f(x)$, если $f(x)$ определено, и $t_T(x)$ не определено, если $f(x)$ не определено.
\end{definition}

\begin{definition}
    \textit{Ленточной сложностью} машины $T$ называется функция $s_T(x)$, значение которой равно числу ячеек машины $T$, используемых при вычислении значения $f(x)$, и $s_T(x)$ не определено, если $f(x)$ не определено. 
\end{definition}

\begin{definition}
    Говорят, что машина Тьюринга $T$ имеет \textit{полиномиальную временную сложность} $P(n)=n^k$ (или <<время работы ограничено полиномом $P(n)$>>), если, обрабатывая вход $w$ длины $n$, $T$ делает не более $P(n)$ переходов и останавливается независимо от того, допущен вход или нет.
\end{definition}