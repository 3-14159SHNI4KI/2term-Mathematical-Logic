\begin{definition}
    Любая формула $\Phi$ представляется в следующем виде:
    $$\Phi = (K_1 x_1)\ldots(K_n x_n)\Uppsi,$$
    
    где $K_1,\ldots,K_n$ -- некоторые кванторы и $\Uppsi$ -- формула без кванторов.

    Таким образом, каждая формула $\Phi$ логически равносильна формуле $(K_1 x_1)\ldots(K_n x_n)\Uppsi$, в которой все кванторы стоят в самом начале формулы и которая называется \textit{предваренной} или \textit{пренексной нормальной формой} (cокращенно ПНФ) формулы $\Phi$.

    При этом последовательность кванторов $(K_1 x_1)\ldots(K_n x_n)$ называется \textit{кванторной приставкой} и формула $\Uppsi$ называется \textit{конъюнктивным ядром} формулы $\Phi$.
\end{definition}

\subsection*{Алгоритм приведения формулы Ф к ПНФ:}
\begin{enumerate}
    \item Преобразуем формулу $\Phi$ в эквивалентную ей 
    формулу $\Phi'$, которая не содержит импликации и 
    эквивалентности и в которой отрицание 
    действует только на элементарные формулы;

    \item В $\Phi'$ все кванторы последовательно выносим 
    вперед по теореме 5, при этом кванторы 
    общности $(\forall x)$ выносятся из конъюнкции и 
    кванторы существования $(\exists x)$ выносятся из 
    дизъюнкции, а для выноса кванторов общности 
    $(\forall x)$ из дизъюнкции и кванторов существования 
    $(\exists x)$ из конъюнкции переименовываем связанные 
    переменные $x$ в новые переменные $y$, которые не 
    входят в рассматриваемую формулу.
    
\end{enumerate}