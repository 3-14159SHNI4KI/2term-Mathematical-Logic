\begin{definition}
    Определение распознавательной задачи:

    Имеется множество объектов $X$ и определенное подмножество $P\subset X$, требуется найти эффективную процедуру (т.е. алгоритм), с помощью которой для любого $x\in X$ можно определить $x\in P$ или $x \notin P$.

    При этом распознавательная задача называется \textit{алгоритмически разрешимой} или \textit{алгоритмически неразрешимой} в зависимости от того, имеется или нет алгоритм решения этой задачи.
\end{definition}

Конструктивные объекты любого множества $X$ можно \textit{кодировать} словами конечного множества $\Sigma$ (например, состоящего из двоичных символов 0 и 1) с помощью взаимно"=однозначного отображения $f: X\rightarrow\Sigma^*$, где $\Sigma^*$ -- множество всех слов над алфавитом $\Sigma$.

В результате распознавательная задача формулируется следующим универсальным образом:

Имеется множество слов $W\subset\Sigma^*$ над некоторым алфавитом $\Sigma$ и определенный язык $L\subset W$, требуется найти эффективную процедуру (т.е. алгоритм), с помощью которой для любого слова $w\in W$ можно определить $w\in L$ или $w\notin L$.

\begin{definition}
    Язык $L$ называется \textit{разрешимым} (или \textit{рекурсивным}), если существует такая машина Тьюринга $T$, что для любого слова $w\in W$ выполняются условия:
    \begin{enumerate}
        \item Если $w\in L$, то при входе $w$ машина $T$ попадает в заключительное состояние, останавливается и выдает значение $T(w)=1$
        \item Если $w\notin L$, то при входе $w$ машина $T$ попадает в заключительное состояние, останавливается и выдает значение $T(w)=0$
    \end{enumerate}

    Такие машины соответствуют понятию <<алгоритма>> и применяются при решении \textit{распознавательных задач} типа <<да/нет>>.
\end{definition}

Множество всех разрешимых языков будет обозначать \textbf{R} (от Recursive).

\textbf{Свойства: } дополнения, конечные пересечения и конечные объединения разрешимых языков являются разрешимыми языками.

\textbf{Примеры разрешимых языков}
\begin{itemize}
    \item Пустой язык
    \item Множество всех строк
    \item Конечные множества
    \item Множество четных чисел
    \item Множество простых чисел
    \item Множество рациональных чисел, меньших числа $e$
    \item Множество $n$, при которых в числе $\pi$ есть $n$ девяток подряд
\end{itemize}

\begin{definition}
    Язык $L$ называется \textit{полуразрешимым} или \textit{перечислимым}, если существует такая машина Тьюринга $T$, что
    $$L=L(T) = \{w\in \Sigma^*:T(w)=1\},$$
    т.е. при входе $w\in L$ машина $T$ попадает в заключительное состояние, останавливается и выдает значение $T(w) = 1$, а при входе $w\notin L$ машина Тьюринга $T$ не дает никакого результата.
\end{definition}

Множество всех полуразрешимых языков будем обозначать \textbf{RE} (от Recursive Enumerable).

\begin{lemma}
    Существуют неразрешимые языки, посколько алгоритмов счетное число, а языков несчетное.

    Аналогично можно доказать, что существуют языки, не являющиеся полуразрешимыми.
\end{lemma}

\begin{theorem}
    Существуют полуразрешимые неразрешимые языки, т.е. полуразрешимые языки, которые не могут быть разрешимым никаким алгоритмом, т.е. выполняется свойство \textbf{R} $\not\subset$ \textbf{RE}.
\end{theorem}

\subsection*{Теорема Поста?????}
(из википедии)

Теорема Поста — теорема теории вычислимости о рекурсивно перечислимых множествах. 

Если множество $M$ и его дополнение $\overline{M}$ в множестве натуральных чисел $\mathbb{N} $ рекурсивно перечислимы, то множества $M$ и $\overline{M}$ разрешимы.

(\textit{Дополнением множества $M$ до универсального множества $U$}, или просто \textit{дополнением}, называется множество всех элементов множества $U$, не принадлежащих $M$. Обозн. $\overline{M}.$)